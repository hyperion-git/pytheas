\documentclass[11pt,a4paper]{article}

\usepackage[utf8]{inputenc}
\usepackage[T1]{fontenc}
\usepackage{lmodern}
\usepackage{amsmath,amssymb}
\usepackage{booktabs}
\usepackage{geometry}
\usepackage{hyperref}
\usepackage{xcolor}
\usepackage{cancel}

\geometry{margin=2.5cm}
\hypersetup{colorlinks=true, linkcolor=blue!60!black, urlcolor=blue!60!black}

\newcommand{\vb}[1]{\mathbf{#1}}
\newcommand{\vu}[1]{\vb{e}_{\vb{#1}}}

\title{Why a Tilted Sensor Cannot Measure\\Direct Gravitational Pull}
\author{}
\date{}

\begin{document}
\maketitle

\section*{The Claim}

\begin{quote}
A sensor tilted relative to the local vertical can detect the direct
gravitational pull of the Sun (${\sim}\,6 \times 10^{-3}$~m/s$^2$) or the
Moon (${\sim}\,3.3 \times 10^{-5}$~m/s$^2$) as a time-varying signal,
because the projection of the gravitational acceleration onto the sensor
axis changes as these bodies move across the sky.
\end{quote}

\noindent
This is incorrect. The derivation below shows, in a point-mass model with
no approximations beyond Newtonian mechanics, that the direct pull of
\emph{every} external body cancels exactly against the corresponding
acceleration of the laboratory, leaving only the tidal residual ---
regardless of sensor orientation.


\section{Setup}

Four point masses in an inertial frame:

\begin{center}
\begin{tabular}{lll}
\toprule
Body & Mass & Position \\
\midrule
Sun       & $M_S$ & $\vb{R}_S(t)$ \\
Moon      & $M_M$ & $\vb{R}_M(t)$ \\
Earth     & $M_E$ & $\vb{R}_E(t)$ \\
Test mass & $m$   & $\vb{x}(t)$ \\
\bottomrule
\end{tabular}
\end{center}

The test mass sits on Earth's surface and is connected to the sensor by a
spring (or equivalent restoring mechanism). The sensor frame is rigidly
attached to the Earth.

The sensor reads the \textbf{spring force} $\vb{F}_{\text{spring}}$ ---
the non-gravitational force required to keep the test mass co-moving with
the lab. This is the only quantity accessible to experiment; gravity itself
is not directly measurable (equivalence principle).


\section{Equation of Motion in the Inertial Frame}

Newton's second law for the test mass:
\begin{equation}
m\,\ddot{\vb{x}}
  = \underbrace{-\frac{G M_E\, m}{\lvert\vb{x} - \vb{R}_E\rvert^3}
    (\vb{x} - \vb{R}_E)}_{\text{Earth}}
  \;\underbrace{-\frac{G M_S\, m}{\lvert\vb{x} - \vb{R}_S\rvert^3}
    (\vb{x} - \vb{R}_S)}_{\text{Sun}}
  \;\underbrace{-\frac{G M_M\, m}{\lvert\vb{x} - \vb{R}_M\rvert^3}
    (\vb{x} - \vb{R}_M)}_{\text{Moon}}
  \;+\; \vb{F}_{\text{spring}}
\label{eq:inertial}
\end{equation}


\section{Equation of Motion for Earth's Center}

Earth's center of mass is accelerated by both external bodies:
\begin{equation}
\ddot{\vb{R}}_E
  = -\frac{G M_S}{\lvert\vb{R}_E - \vb{R}_S\rvert^3}
    (\vb{R}_E - \vb{R}_S)
  \;-\; \frac{G M_M}{\lvert\vb{R}_E - \vb{R}_M\rvert^3}
    (\vb{R}_E - \vb{R}_M)
\label{eq:earth}
\end{equation}

This is Earth's acceleration toward the Sun and Moon combined. Every object
attached to the Earth --- including the sensor housing, the mounting
bracket, and the reference frame of the measurement --- shares this
acceleration.


\section{Transform to the Earth-Centered Frame}

Define the position of the test mass relative to Earth's center:
\begin{equation}
\vb{r} \equiv \vb{x} - \vb{R}_E
\end{equation}
and the geocentric positions of the external bodies:
\begin{equation}
\vb{R} \equiv \vb{R}_S - \vb{R}_E, \qquad
\vb{D} \equiv \vb{R}_M - \vb{R}_E
\end{equation}

The relative acceleration is
$\ddot{\vb{r}} = \ddot{\vb{x}} - \ddot{\vb{R}}_E$. Substituting from
Eqs.~\eqref{eq:inertial} and~\eqref{eq:earth}, each external body~$B$
produces a term of the form
\begin{equation}
-\frac{G M_B\,m}{\lvert\vb{x} - \vb{R}_B\rvert^3}(\vb{x} - \vb{R}_B)
\;+\; \frac{G M_B\,m}{\lvert\vb{R}_E - \vb{R}_B\rvert^3}(\vb{R}_E - \vb{R}_B)
\end{equation}
where $B \in \{S, M\}$. Rewriting in geocentric variables:

\begin{equation}
\boxed{
m\,\ddot{\vb{r}}
  = -\frac{G M_E\, m}{\lvert\vb{r}\rvert^3}\,\vb{r}
  \;+\; G M_S\, m \left[
    \frac{\vb{R} - \vb{r}}{\lvert\vb{R} - \vb{r}\rvert^3}
    - \frac{\vb{R}}{\lvert\vb{R}\rvert^3}
  \right]
  \;+\; G M_M\, m \left[
    \frac{\vb{D} - \vb{r}}{\lvert\vb{D} - \vb{r}\rvert^3}
    - \frac{\vb{D}}{\lvert\vb{D}\rvert^3}
  \right]
  \;+\; \vb{F}_{\text{spring}}
}
\label{eq:geocentric}
\end{equation}

\subsection*{Exact cancellation of the direct pull}

Consider the Sun's contribution to Eq.~\eqref{eq:geocentric}. It arose
from subtracting Earth's acceleration (Eq.~\ref{eq:earth}) from the test
mass acceleration (Eq.~\ref{eq:inertial}). Tracing the Sun's terms
through the subtraction:

\medskip
\noindent
\textbf{From the test mass} (Eq.~\ref{eq:inertial}), the Sun contributes
an acceleration:
\begin{equation}
-\frac{G M_S}{\lvert\vb{x} - \vb{R}_S\rvert^3}
  (\vb{x} - \vb{R}_S)
\label{eq:sun_test}
\end{equation}

\noindent
\textbf{From Earth's center} (Eq.~\ref{eq:earth}), subtracted via
$\ddot{\vb{r}} = \ddot{\vb{x}} - \ddot{\vb{R}}_E$:
\begin{equation}
+\frac{G M_S}{\lvert\vb{R}_E - \vb{R}_S\rvert^3}
  (\vb{R}_E - \vb{R}_S)
\label{eq:sun_earth}
\end{equation}

\noindent
Substituting $\vb{x} - \vb{R}_S = \vb{r} - \vb{R}$ and
$\vb{R}_E - \vb{R}_S = -\vb{R}$, their sum becomes:
\begin{equation}
G M_S \underbrace{\left[
  \frac{\vb{R} - \vb{r}}{\lvert\vb{R} - \vb{r}\rvert^3}
  - \frac{\vb{R}}{R^3}
\right]}_{\displaystyle\equiv\;\vb{f}(\vb{r})}
\label{eq:bracket}
\end{equation}

This expression $\vb{f}(\vb{r})$ is the difference of the Sun's
gravitational field evaluated at two points: the test mass location
($\vb{r}$ from Earth's center) and Earth's center itself ($\vb{r}=0$).
By construction:
\begin{equation}
\vb{f}(\vb{0})
  = \frac{\vb{R}}{R^3} - \frac{\vb{R}}{R^3}
  = \vb{0}
  \qquad\text{(exactly)}
\label{eq:exact_cancel}
\end{equation}

This is the exact cancellation. The ``direct pull'' ---
$G M_S\,\vb{R}/R^3 = (G M_S / R^2)\,\vu{R}$ --- is the value of
the Sun's field at Earth's center. It appears with opposite signs in
Eqs.~\eqref{eq:sun_test} and~\eqref{eq:sun_earth} and cancels
identically, without approximation. What remains,
$\vb{f}(\vb{r}) = \vb{f}(\vb{r}) - \vb{f}(\vb{0})$, is purely the
\emph{variation} of the field across the distance~$\vb{r}$ --- the tidal
acceleration.

The identical cancellation occurs for the Moon, with $\vb{D}$ replacing
$\vb{R}$. The result generalises to any number of external bodies: for
each, the direct pull is common to test mass and Earth, and subtracts out
exactly.


\subsection*{Taylor expansion: identifying the surviving term}

To determine the magnitude and structure of what survives, expand
$\vb{f}(\vb{r})$ for a generic body at geocentric position~$\vb{R}$
(with $R = \lvert\vb{R}\rvert$ and $\vu{R} = \vb{R}/R$) in powers of
the small parameter $\lvert\vb{r}\rvert / R$.

\medskip\noindent
Start with the inverse cube:
\begin{equation}
\lvert\vb{R} - \vb{r}\rvert^2
  = R^2\!\left(1 - 2\,\frac{\vu{R} \cdot \vb{r}}{R}
    + \frac{r^2}{R^2}\right)
\end{equation}
so, defining $\epsilon = 2\,\vu{R}\cdot\vb{r}/R - r^2\!/R^2 = O(r/R)$:
\begin{equation}
\frac{1}{\lvert\vb{R} - \vb{r}\rvert^3}
  = \frac{1}{R^3}(1 - \epsilon)^{-3/2}
  = \frac{1}{R^3}\!\left(1 + \tfrac{3}{2}\epsilon
    + O(\epsilon^2)\right)
  = \frac{1}{R^3}\!\left(1
    + \frac{3\,\vu{R}\cdot\vb{r}}{R}
    + O\!\left(\frac{r^2}{R^2}\right)\right)
\end{equation}

\noindent
Multiply by $(\vb{R} - \vb{r})$ and keep terms through first order in
$r/R$:
\begin{align}
\frac{\vb{R} - \vb{r}}{\lvert\vb{R} - \vb{r}\rvert^3}
  &= \frac{1}{R^3}\!\left(1 + \frac{3\,\vu{R}\cdot\vb{r}}{R}\right)
     (\vb{R} - \vb{r})
     + O\!\left(\frac{r^2}{R^4}\right) \nonumber \\[4pt]
  &= \frac{1}{R^3}\Big[
       \vb{R}
       + 3(\vu{R}\cdot\vb{r})\,\vu{R}
       - \vb{r}
     \Big]
     + O\!\left(\frac{r^2}{R^4}\right)
\label{eq:expansion}
\end{align}
where the cross term $-3(\vu{R}\cdot\vb{r})\,\vb{r}/R$ is $O(r^2/R)$
and has been absorbed into the remainder.

\noindent
Now subtract $\vb{R}/R^3$:
\begin{equation}
\underbrace{
\frac{\vb{R} - \vb{r}}{\lvert\vb{R} - \vb{r}\rvert^3}
  - \frac{\vb{R}}{R^3}}_{\text{Eq.~\eqref{eq:bracket}}}
  = \frac{1}{R^3}\Big[
      \cancel{\vb{R}}
      + 3(\vu{R}\cdot\vb{r})\,\vu{R}
      - \vb{r}
      - \cancel{\vb{R}}
    \Big]
    + O\!\left(\frac{r^2}{R^4}\right)
\label{eq:cancel}
\end{equation}

The $\vb{R}/R^3$ terms --- which represent the \textbf{direct
gravitational pull} $GM/R^2$ in the direction $\vu{R}$ --- cancel
identically. What survives is the quadrupole (tidal) term:
\begin{equation}
\boxed{
\vb{a}_{\text{tidal}}
  = \frac{GM}{R^3}\Big[
      3\,(\vu{R}\cdot\vb{r})\,\vu{R} - \vb{r}
    \Big]
  + O\!\left(\frac{GM\,r^2}{R^4}\right)
}
\label{eq:tidal_approx}
\end{equation}

This is suppressed by a factor of $r/R$ relative to the direct pull ---
the ratio of Earth's radius to the distance to the external body.
For the Sun, $r/R \approx 4.3 \times 10^{-5}$; for the Moon,
$r/D \approx 1.7 \times 10^{-2}$.

Applying Eq.~\eqref{eq:tidal_approx} to each body:
\begin{align}
\vb{a}_{\text{tidal}}^{\text{Sun}}
  &= G M_S \left[
    \frac{\vb{R} - \vb{r}}{\lvert\vb{R} - \vb{r}\rvert^3}
    - \frac{\vb{R}}{\lvert\vb{R}\rvert^3}
  \right]
  \;\approx\;
  \frac{G M_S}{R^3}\Big[
    3\,(\vu{R}\cdot\vb{r})\,\vu{R} - \vb{r}
  \Big]
\label{eq:tidal_sun} \\[6pt]
\vb{a}_{\text{tidal}}^{\text{Moon}}
  &= G M_M \left[
    \frac{\vb{D} - \vb{r}}{\lvert\vb{D} - \vb{r}\rvert^3}
    - \frac{\vb{D}}{\lvert\vb{D}\rvert^3}
  \right]
  \;\approx\;
  \frac{G M_M}{D^3}\Big[
    3\,(\vu{D}\cdot\vb{r})\,\vu{D} - \vb{r}
  \Big]
\label{eq:tidal_moon}
\end{align}

To summarise:
\begin{itemize}
\item The \textbf{exact cancellation} (Eq.~\ref{eq:exact_cancel}) is
  algebraic: the direct pull drops out of $\vb{f}(\vb{r})$ because
  $\vb{f}(\vb{0}) = \vb{0}$ by construction. No expansion is needed.
\item The \textbf{zeroth-order} term in the Taylor expansion
  ($\vb{R}/R^3$, Eq.~\ref{eq:cancel}) confirms this: it cancels
  identically against the subtracted $\vb{R}/R^3$.
\item The \textbf{first-order} term ($\propto r/R^3$,
  Eq.~\ref{eq:tidal_approx}) is the leading surviving contribution ---
  the tidal acceleration with quadrupolar structure: stretching along the
  Earth--body axis, compression perpendicular, trace-free.
\end{itemize}


\section{What the Sensor Reads}

The sensor reads $\vb{F}_{\text{spring}}$, which already appears in
Eq.~\eqref{eq:geocentric}. To isolate it, we solve that equation for
$\vb{F}_{\text{spring}}$ by specifying the acceleration
$\ddot{\vb{r}}$ of the test mass.

The test mass is stationary in the lab frame, which co-rotates with
Earth at angular velocity~$\boldsymbol{\omega}$. In the non-rotating
geocentric frame used in Eq.~\eqref{eq:geocentric}, this corresponds
to circular motion with centripetal acceleration:
\begin{equation}
\ddot{\vb{r}}
  = \boldsymbol{\omega} \times (\boldsymbol{\omega} \times \vb{r})
  = -\omega^2\,\vb{r}_\perp
\label{eq:centripetal}
\end{equation}
where $\vb{r}_\perp$ is the component of~$\vb{r}$ perpendicular to the
rotation axis (pointing outward from that axis).

Substituting Eq.~\eqref{eq:centripetal} into the boxed
equation~\eqref{eq:geocentric} and solving for
$\vb{F}_{\text{spring}}$:
\begin{align}
\cancel{m}\,\boldsymbol{\omega} \times (\boldsymbol{\omega} \times \vb{r})
  &= -\frac{G M_E\,\cancel{m}}{\lvert\vb{r}\rvert^3}\,\vb{r}
  + \cancel{m}\,\vb{a}_{\text{tidal}}^{\text{Sun}}
  + \cancel{m}\,\vb{a}_{\text{tidal}}^{\text{Moon}}
  + \frac{\vb{F}_{\text{spring}}}{\cancel{m}\,} \nonumber \\[6pt]
\vb{F}_{\text{spring}}
  &= m\left[
    \frac{G M_E}{\lvert\vb{r}\rvert^3}\,\vb{r}
    + \boldsymbol{\omega} \times (\boldsymbol{\omega} \times \vb{r})
    - \vb{a}_{\text{tidal}}^{\text{Sun}}
    - \vb{a}_{\text{tidal}}^{\text{Moon}}
  \right]
\label{eq:spring}
\end{align}

The three contributions have clear physical meaning:
\begin{itemize}
\item $G M_E\,\vb{r}/\lvert\vb{r}\rvert^3$: Earth's gravitational
  acceleration (the spring supports the test mass against this).
\item $\boldsymbol{\omega}\times(\boldsymbol{\omega}\times\vb{r})
  = -\omega^2\vb{r}_\perp$: centripetal acceleration from Earth's
  rotation, which \emph{reduces} the required spring force (objects
  weigh less at the equator).
\item $-\vb{a}_{\text{tidal}}$: the tidal perturbations from
  Eqs.~\eqref{eq:tidal_sun}--\eqref{eq:tidal_moon}, acting as small
  corrections.
\end{itemize}

The sensor projects this onto its measurement axis $\vu{n}$:
\begin{equation}
g_{\text{measured}}
  = \frac{\vb{F}_{\text{spring}}}{m} \cdot \vu{n}
\end{equation}

\textbf{Neither the direct solar acceleration} $G M_S / R^2$ \textbf{nor
the direct lunar acceleration} $G M_M / D^2$ \textbf{appears.} These
terms canceled in Eq.~\eqref{eq:geocentric} before we ever solved for
$\vb{F}_{\text{spring}}$ --- the spring force inherits only the tidal
residuals. This holds for any~$\vu{n}$.


\section{Why Tilting Doesn't Help}

The claim implicitly assumes that tilting the sensor changes the projection
of a body's \emph{direct} pull onto $\vu{n}$:
\begin{equation}
g_{\text{claimed}}^{\text{Sun}} = \frac{G M_S}{R^2}\,(\vu{R} \cdot \vu{n}),
\qquad
g_{\text{claimed}}^{\text{Moon}} = \frac{G M_M}{D^2}\,(\vu{D} \cdot \vu{n})
\quad\longleftarrow\quad \text{these terms do not exist}
\end{equation}

But each was canceled by the identical projection of Earth's acceleration
toward that body onto the same axis. Tilting $\vu{n}$ rotates all
projections equally, because the cancellation is \textbf{vectorial} --- it
holds component by component, in every direction simultaneously.

To make the argument concrete: the sensor housing accelerates at
$\ddot{\vb{R}}_E$ (toward both the Sun and Moon). The test mass, if
released, would also accelerate at $\ddot{\vb{R}}_E$ (plus the tidal
corrections). The spring connecting them sees only the difference --- which
is the sum of the two tidal terms. This is true regardless of the spring's
orientation.


\section{Numerical Comparison}

\subsection*{Sun}

\begin{center}
\begin{tabular}{llr}
\toprule
Quantity & Formula & Value \\
\midrule
Direct pull & $G M_S / R^2$ & $5.93 \times 10^{-3}$~m/s$^2$ \\
Earth's acceleration toward Sun (cancels) & same & $5.93 \times 10^{-3}$~m/s$^2$ \\
\textbf{Tidal residual} & $\sim 2\,G M_S\, r_\oplus / R^3$ & $\mathbf{5.1 \times 10^{-7}}$~\textbf{m/s}$^2$ \\
\bottomrule
\end{tabular}
\end{center}

\begin{equation}
\frac{a_{\text{tidal}}^{\text{Sun}}}{a_{\text{direct}}^{\text{Sun}}}
= \frac{2\,r_\oplus}{R_{\text{Sun}}}
= \frac{2 \times 6.371 \times 10^6}{1.496 \times 10^{11}}
\approx 8.5 \times 10^{-5}
\end{equation}

The direct pull is \textbf{11,700$\boldsymbol{\times}$} larger than the
measurable tidal signal.

\subsection*{Moon}

\begin{center}
\begin{tabular}{llr}
\toprule
Quantity & Formula & Value \\
\midrule
Direct pull & $G M_M / D^2$ & $3.32 \times 10^{-5}$~m/s$^2$ \\
Earth's acceleration toward Moon (cancels) & same & $3.32 \times 10^{-5}$~m/s$^2$ \\
\textbf{Tidal residual} & $\sim 2\,G M_M\, r_\oplus / D^3$ & $\mathbf{1.1 \times 10^{-6}}$~\textbf{m/s}$^2$ \\
\bottomrule
\end{tabular}
\end{center}

\begin{equation}
\frac{a_{\text{tidal}}^{\text{Moon}}}{a_{\text{direct}}^{\text{Moon}}}
= \frac{2\,r_\oplus}{D_{\text{Moon}}}
= \frac{2 \times 6.371 \times 10^6}{3.844 \times 10^{8}}
\approx 3.3 \times 10^{-2}
\end{equation}

The direct pull is \textbf{30$\boldsymbol{\times}$} larger than the
measurable tidal signal. The Moon's ratio is much less extreme than the
Sun's because the Moon is only ${\sim}\,60$ Earth radii away, so the tidal
approximation is coarser --- but the cancellation is still exact.

\subsection*{Summary}

\begin{center}
\begin{tabular}{lllrl}
\toprule
Body & Direct pull & Tidal residual & Ratio & Suppression by \\
\midrule
Sun  & $5.93 \times 10^{-3}$~m/s$^2$ & $5.1 \times 10^{-7}$~m/s$^2$ & 11,700$\times$ & $R_{\text{Sun}} / 2r_\oplus$ \\
Moon & $3.32 \times 10^{-5}$~m/s$^2$ & $1.1 \times 10^{-6}$~m/s$^2$ & 30$\times$ & $D_{\text{Moon}} / 2r_\oplus$ \\
\bottomrule
\end{tabular}
\end{center}

Note the inversion: the Sun's direct pull is $180\times$ stronger than the
Moon's, but its tidal effect is $2.2\times$ \emph{weaker}. This is the
$1/R^3$ vs $1/R^2$ scaling --- the hallmark of a tidal interaction, and a
direct consequence of the cancellation derived above.


\section{The Point-Mass Model Proves the Cancellation}

The crucial point: the cancellation above used \textbf{only} Newton's laws
applied to point masses. No general relativity, no equivalence principle,
no appeal to ``free fall'' as a concept --- just subtracting the equation
of motion of Earth's center from the equation of motion of the test mass.

For each external body~$B$, the term $G M_B / \lvert\vb{R}_B'\rvert^2$
appears with the same sign, same magnitude, and same direction in both
equations, so it drops out identically.

Anyone who computes $G M_S / R^2 \approx 6 \times 10^{-3}$~m/s$^2$ or
$G M_M / D^2 \approx 3.3 \times 10^{-5}$~m/s$^2$ and projects it onto a
sensor axis is computing a quantity that is real (those bodies do pull the
test mass that hard) but \textbf{unobservable} --- because the sensor, the
lab, and the entire Earth are being pulled equally hard in the same
direction. The spring between the test mass and the sensor frame cannot
detect a uniform acceleration field.


\section{Why the Cancellation Works: Universality of Free Fall}

One might object: the test mass~$m$ and the Earth~$M_E$ have very
different masses, so why should the Sun's effect cancel between them?

The answer is that gravitational \textbf{force} is proportional to the
accelerated mass ($F = G M_S\, m / R^2$), so the \textbf{acceleration} is
mass-independent:
\begin{equation}
a = \frac{F}{m} = \frac{G M_S}{R^2}
\end{equation}

This is the equality of inertial and gravitational mass
($m_{\text{inert}} = m_{\text{grav}}$), built into Newton's law of
gravitation. The test mass and Earth's center experience the same
gravitational acceleration from the Sun (to leading order in $r/R$),
despite their mass ratio of ${\sim}\,10^{-25}$.

This is \textbf{not} Newton's third law (action = reaction between two
interacting bodies). It is a deeper property: two \emph{different} bodies
in the \emph{same} external field accelerate identically. The spring
connecting them therefore reads zero from the external field --- only the
spatial \emph{variation} of the field (the tidal gradient) survives.

If any material violated this equality --- if some substance fell faster
than others in the Sun's field --- the direct pull \emph{would} be
measurable as a composition-dependent residual. This is precisely what
Eötvös-type experiments test, achieving constraints at the $10^{-15}$
level. Their null results confirm that the cancellation is exact to
extraordinary precision, and that the tidal residual is all that remains.


\section{Why Converting to Frequency Doesn't Help}

One might attempt to circumvent the cancellation by converting the
acceleration measurement into a frequency measurement: use an
optomechanical oscillator (a mirror on a spring, coupled to a cavity)
whose resonance frequency depends on~$g$, and beat it against a reference
cavity whose frequency depends only on its length. If the Sun's pull
shifts $\omega_{\text{mech}}$ but not $f_{\text{cav}}$, the beat note
should reveal the direct pull.

This fails because the cancellation occurs \emph{before} the readout ---
at the level of what forces exist between the test mass and its mount.

\subsection*{The optomechanical oscillator}

Consider a mirror of mass~$m$ on a spring of constant~$k$, anchored to the
lab ceiling. In an inertial frame, equilibrium requires:
\begin{equation}
k\,x_{\text{eq}}
  = m\,\big(g_{\text{mass}} - a_{\text{anchor}}\big)
\end{equation}
where $g_{\text{mass}}$ is the total gravitational acceleration at the
mirror, and $a_{\text{anchor}}$ is the acceleration of the anchor point
(which co-moves with the Earth). From Section~4:
\begin{equation}
g_{\text{mass}} - a_{\text{anchor}}
  = g_{\text{Earth}} - \omega^2 r_\perp
    + a_{\text{tidal}}^{\text{Sun}}
    + a_{\text{tidal}}^{\text{Moon}}
\end{equation}

The Sun's direct pull appears in both $g_{\text{mass}}$ and
$a_{\text{anchor}}$ and cancels. The equilibrium position --- and
therefore the oscillation frequency $\omega_{\text{mech}} \propto
\sqrt{g_{\text{eff}}/L}$ --- contains only the tidal part of the solar
and lunar fields.

\subsection*{The reference cavity}

A Fabry--P\'{e}rot reference cavity has resonance frequency $f = mc/(2L)$.
The spacer length~$L$ is set by electromagnetic bond forces. The Sun's
uniform gravitational field accelerates every atom in the spacer
equally --- no differential stress, no deformation. Only the tidal
gradient across the cavity length~$L$ would deform it:
\begin{equation}
\frac{\delta L}{L}
  \sim \frac{GM_S}{R^3}\,\frac{L^2}{E/\rho}
  \sim 10^{-23}
\end{equation}
which is unmeasurable.

\subsection*{The beat note}

The beat frequency $\Delta f = f_{\text{mech}} - f_{\text{cav}}$ therefore
contains: tidal effects from the oscillator, minus negligible tidal
deformation of the cavity. The direct pull $GM_S/R^2$ appears in neither
channel.

Converting acceleration to frequency does not change \emph{what physical
quantity} is being measured. The spring (or pendulum, or optomechanical
cavity) connects two objects --- test mass and housing --- that share the
same orbital acceleration. It can only measure the differential force
between its endpoints, regardless of whether the readout is a
displacement, a voltage, or a frequency.


\section{What Frequency Measurements \emph{Can} Detect}

While no local mechanical measurement can access the direct pull, the
gravitational \textbf{potential} (as opposed to the force) can be
detected through a fundamentally different channel: the gravitational
redshift.

An atomic clock's tick rate depends on the spacetime metric:
\begin{equation}
\frac{d\tau}{dt} = \sqrt{-g_{00}}
  \approx 1 + \frac{\Phi}{c^2}
\end{equation}
where $\Phi$ is the gravitational potential. This is not a force between
two co-accelerating parts --- it is a property of spacetime itself.

However, the equivalence principle guarantees that \emph{all} local
physics is shifted equally by the potential. A single clock in the Sun's
field has nothing local to compare against. To detect the potential, one
needs a \textbf{non-local} reference --- a frequency source outside the
freely-falling frame.

\subsection*{Clock vs.\ distant reference (pulsar)}

The Sun's potential at Earth is
$\Phi_\odot = -GM_S/R \approx -8.9 \times 10^{8}$~m$^2$/s$^2$.
Because Earth's orbit is elliptical ($e \approx 0.0167$), this varies
annually:
\begin{equation}
\frac{\Delta f}{f}
  = \frac{GM_S}{c^2}
    \left(\frac{1}{R_{\text{peri}}} - \frac{1}{R_{\text{aph}}}\right)
  \approx 3.3 \times 10^{-10}
\end{equation}

This is the \textbf{full, direct} solar potential --- not tidal, not
suppressed by~$r_\oplus/R$. Pulsar timing arrays detect exactly this
effect as the ``Einstein delay,'' with an annual amplitude of
${\sim}\,1.66$~ms.

\subsection*{Two terrestrial clocks}

Two clocks at different locations on Earth see different solar
potentials, but the difference is only the tidal potential:
\begin{equation}
\frac{\delta f}{f}
  \sim \frac{GM_S\,r_\oplus^2}{R^3\,c^2}
  \sim 10^{-17}
\end{equation}

Modern optical lattice clocks ($\Delta f/f \sim 10^{-18}$) are just
reaching this regime.

\subsection*{Hierarchy of measurements}

\begin{center}
\begin{tabular}{llll}
\toprule
Method & Measures & Sun contribution & Magnitude \\
\midrule
Accelerometer       & $-\nabla\Phi$ (force)
  & tidal only        & $5 \times 10^{-7}$~m/s$^2$ \\
Optomech.\ vs.\ ref.\ cavity & $-\nabla\Phi$ (force as frequency)
  & tidal only        & $5 \times 10^{-7}$~m/s$^2$ \\
Two local clocks    & $\Delta\Phi$ (potential difference)
  & tidal potential   & $\Delta f/f \sim 10^{-17}$ \\
Clock vs.\ pulsar   & $\Phi$ (absolute potential)
  & \textbf{full direct} & $\Delta f/f \sim 3 \times 10^{-10}$ \\
\bottomrule
\end{tabular}
\end{center}

The dividing line: any instrument where two parts are co-accelerating sees
the cancellation. To escape it, one end of the measurement must be outside
the freely-falling frame --- a non-local comparison of proper time flow
rates at different locations in the potential.


\section{What IS Observable by Tilting}

Tilting the sensor does reveal genuinely interesting physics, but at the
tidal scale (${\sim}\,10^{-7}$~m/s$^2$), not at the direct-pull scale
(${\sim}\,10^{-3}$~m/s$^2$):

\begin{itemize}
\item \textbf{Horizontal tidal acceleration.} A vertical sensor sees only
  the vertical component of $\vb{a}_{\text{tidal}}$. A tilted sensor picks
  up horizontal components, which have comparable magnitude but different
  time dependence.

\item \textbf{Angular separation of static and tidal signals.} Normal
  gravity is purely vertical (projects as $g_0 \cos\theta$), while the
  tidal vector has East/North/Up components with independent angular
  signatures. Scanning the sensor azimuth at $\theta = 90°$ gives a pure
  tidal signal with zero static background.

\item \textbf{Reconstruction of the full tidal vector.} Three non-coplanar
  measurements at one instant determine all four unknowns ($g_0$, $a_E$,
  $a_N$, $a_U$) without a time series.
\end{itemize}

These effects are real, measurable, and interesting --- but they are four
orders of magnitude smaller than the claimed signal.

\end{document}
