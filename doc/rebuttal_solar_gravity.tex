\documentclass[aps,prx,onecolumn,superscriptaddress,nofootinbib]{revtex4-2}

\usepackage{amsmath,amssymb,amsthm}
\usepackage{booktabs}
\usepackage{graphicx}
\usepackage{hyperref}
\usepackage{xcolor}

\hypersetup{colorlinks=true, linkcolor=blue!60!black, urlcolor=blue!60!black}

\newtheorem{theorem}{Theorem}

\newcommand{\vb}[1]{\mathbf{#1}}
\newcommand{\vu}[1]{\vb{e}_{\vb{#1}}}

\begin{document}

\title{Falling Together: Why Springs Can't See the Sun}
\author{Alexander Friedrich}
\affiliation{DLR Institute of Quantum Technologies, Wilhelm-Runge-Str.\ 10, 89081 Ulm, Germany}
\date{\today}

\begin{abstract}
A common claim holds that a spring-based sensor can detect the
direct gravitational pull of the Sun
(${\sim}\,6 \times 10^{-3}$~m/s$^2$) or the Moon
(${\sim}\,3.3 \times 10^{-5}$~m/s$^2$) as a time-varying signal.
This is incorrect. We prove the following:
\emph{The spring force on a test mass held at rest near the Earth
contains no contribution from the direct gravitational pull
$GM/R^2$ of any external body; the direct pull cancels exactly
against the free-fall acceleration of the laboratory, leaving only
the tidal residual of order $GM\,r_\oplus/R^3$ --- regardless of
sensor orientation.}
The proof uses only Newtonian point-mass mechanics in an inertial
frame where no terms are hidden. The result is then transformed to
the rotating laboratory frame, where the only surviving terms ---
Earth's gravity, centrifugal correction, and tidal residuals ---
are fully consistent with standard terrestrial gravimetry.
\end{abstract}

\maketitle


\section{Introduction}
\label{sec:frames}

Can a spring-based accelerometer measure the Sun's direct
gravitational pull on its test mass? This note shows that the
answer is \textbf{no} --- not because the pull is too small, but
because the equivalence principle forbids it. The Sun pulls
the sensor housing and the test mass by exactly the same amount;
the spring connecting them cannot register a force that acts
identically on both ends.

The difficulty in seeing this lies in the choice of reference
frame. A laboratory on Earth's surface is non-inertial for two
reasons:

\begin{enumerate}
\item \textbf{Earth rotates}, producing centrifugal and Coriolis
  pseudo-accelerations (${\sim}\,0.03$~m/s$^2$ at the equator).

\item \textbf{Earth freely falls toward the Sun and Moon.} The
  sensor housing is rigidly attached to the Earth's crust; it
  shares the Earth's free-fall acceleration toward the Sun
  ($GM_S/R^2 \approx 5.93 \times 10^{-3}$~m/s$^2$) and Moon
  ($GM_M/D^2 \approx 3.32 \times 10^{-5}$~m/s$^2$) exactly.
  The test mass, connected only by a spring, is the one body
  free to deviate --- but gravity accelerates it by the same
  amount, so the spring registers no difference.
\end{enumerate}

\noindent
The second point is the origin of the cancellation. When one
writes equations of motion in the lab frame, the free-fall
acceleration toward the Sun has already been subtracted from every
term. A spring-based sensor physically implements this
subtraction: both ends share the same gravitational acceleration,
so the spring reads zero from that field. Only the tiny
\emph{gradient} of the field across the sensor --- the tidal
term --- survives.

The same principle applies to quantum systems: Asenbaum, Overstreet,
and Kasevich~\cite{asenbaum2024} have shown that matter-wave
interferometers and optical clocks are likewise insensitive to
uniform gravitational fields, with all observable effects arising
from non-gravitational accelerations. The present paper arrives
at the same conclusion from the classical side, addresses the
specific misconception about tilted spring sensors, and provides
a detailed coordinate-system analysis.

To make this cancellation explicit rather than hidden, we begin in
a true inertial frame and carry every term (Sec.~\ref{sec:proof}).
We then transform to the rotating laboratory
(Sec.~\ref{sec:lab}), project onto an arbitrary sensor axis
(Sec.~\ref{sec:orientation}), and compare the magnitudes of all
surviving terms (Sec.~\ref{sec:observable}).
Section~\ref{sec:ep} places the result in the context of the
equivalence principle.


\section{The Proof}
\label{sec:proof}

\subsection{Setup}

Four point masses in an inertial frame (Fig.~\ref{fig:setup}).
The test mass is connected to the sensor by a spring (or equivalent
restoring mechanism). The sensor housing is rigidly attached to the
Earth. The sensor reads the spring force~$\vb{F}_{\text{spring}}$ ---
the non-gravitational force required to keep the test mass co-moving
with the lab.

\begin{figure}[h]
\centering
\begin{minipage}[c]{0.30\columnwidth}
\centering
\begin{tabular}{lll}
\toprule
Body & Mass & Position \\
\midrule
Sun       & $M_S$ & $\vb{R}_S(t)$ \\
Moon      & $M_M$ & $\vb{R}_M(t)$ \\
Earth     & $M_E$ & $\vb{R}_E(t)$ \\
Test mass & $m$   & $\vb{x}(t)$ \\
\bottomrule
\end{tabular}
\end{minipage}%
\hfill
\begin{minipage}[c]{0.67\columnwidth}
\centering
\includegraphics[width=\linewidth]{figures/setup_schematic.pdf}
\end{minipage}
\caption{Four-body setup. \emph{Left}: masses and inertial-frame
positions. \emph{Right}: schematic geometry. Gray dashed arrows:
position vectors $\vb{R}_S$, $\vb{R}_E$, $\vb{R}_M$,
and~$\vb{x}$ from the inertial origin. Red solid arrows:
geocentric vectors $\vb{R} = \vb{R}_S - \vb{R}_E$,
$\vb{D} = \vb{R}_M - \vb{R}_E$, and
$\vb{r} = \vb{x} - \vb{R}_E$. The spring connects the test
mass~$m$ to the Earth-fixed sensor housing.}
\label{fig:setup}
\end{figure}

\subsection{Test mass equation of motion}

In the inertial frame, Newton's second law for the test mass:
\begin{equation}
m\,\ddot{\vb{x}}
  = -\frac{G M_E\, m}{\lvert\vb{x} - \vb{R}_E\rvert^3}
    (\vb{x} - \vb{R}_E)
  \;-\frac{G M_S\, m}{\lvert\vb{x} - \vb{R}_S\rvert^3}
    (\vb{x} - \vb{R}_S)
  \;-\frac{G M_M\, m}{\lvert\vb{x} - \vb{R}_M\rvert^3}
    (\vb{x} - \vb{R}_M)
  \;+\; \vb{F}_{\text{spring}}
\label{eq:test}
\end{equation}

\subsection{Earth's equation of motion}

Earth's center accelerates toward the Sun and Moon:
\begin{equation}
\ddot{\vb{R}}_E
  = -\frac{G M_S}{\lvert\vb{R}_E - \vb{R}_S\rvert^3}
    (\vb{R}_E - \vb{R}_S)
  \;-\; \frac{G M_M}{\lvert\vb{R}_E - \vb{R}_M\rvert^3}
    (\vb{R}_E - \vb{R}_M)
\label{eq:earth}
\end{equation}

Every object rigidly attached to the Earth --- including the sensor
housing --- shares this acceleration.

\subsection{Subtract to get the relative motion}

Define $\vb{r} \equiv \vb{x} - \vb{R}_E$ and the geocentric positions
$\vb{R} \equiv \vb{R}_S - \vb{R}_E$,
$\vb{D} \equiv \vb{R}_M - \vb{R}_E$. Then
$\ddot{\vb{r}} = \ddot{\vb{x}} - \ddot{\vb{R}}_E$ gives:
\begin{equation}
m\,\ddot{\vb{r}}
  = -\frac{G M_E\, m}{\lvert\vb{r}\rvert^3}\,\vb{r}
  \;+\; G M_S\, m \left[
    \frac{\vb{R} - \vb{r}}{\lvert\vb{R} - \vb{r}\rvert^3}
    - \frac{\vb{R}}{R^3}
  \right]
  \;+\; G M_M\, m \left[
    \frac{\vb{D} - \vb{r}}{\lvert\vb{D} - \vb{r}\rvert^3}
    - \frac{\vb{D}}{D^3}
  \right]
  \;+\; \vb{F}_{\text{spring}}
\label{eq:relative}
\end{equation}

\subsection{The cancellation}

Each bracketed term has the form
\begin{equation}
\vb{f}(\vb{r})
  = \frac{\vb{R} - \vb{r}}{\lvert\vb{R} - \vb{r}\rvert^3}
  - \frac{\vb{R}}{R^3}
\end{equation}

This is the Sun's gravitational field at the test mass \emph{minus} the
field at Earth's center. At $\vb{r} = \vb{0}$:
\begin{equation}
\vb{f}(\vb{0})
  = \frac{\vb{R}}{R^3} - \frac{\vb{R}}{R^3}
  = \vb{0}
  \qquad\text{(exactly)}
\label{eq:cancel}
\end{equation}

The direct pull --- $GM_S\,\vb{R}/R^3$, the uniform field that
accelerates the entire Earth at $5.93 \times 10^{-3}$~m/s$^2$ ---
appears with equal magnitude and opposite sign in
Eqs.~\eqref{eq:test} and~\eqref{eq:earth}, and cancels identically.
No approximation is involved. The same holds for the Moon with $\vb{D}$
replacing~$\vb{R}$.

What survives is $\vb{f}(\vb{r}) - \vb{f}(\vb{0})$: the
\emph{variation} of the gravitational field across the
baseline~$\vb{r}$ --- the \textbf{tidal acceleration}. To leading order
in $r/R$:
\begin{equation}
\vb{a}_{\text{tidal}}
  = \frac{GM}{R^3}\Big[
    3\,(\vu{R}\cdot\vb{r})\,\vu{R} - \vb{r}
  \Big]
  + O\!\left(\frac{GM\,r^2}{R^4}\right)
\label{eq:tidal}
\end{equation}

This is suppressed by a factor $r/R$ relative to the direct pull.

\subsection{Derivation of Eq.~\eqref{eq:tidal}}

Expand $\lvert\vb{R} - \vb{r}\rvert^{-3}$ for $r \ll R$. Write
\begin{equation}
\lvert\vb{R} - \vb{r}\rvert^2
  = R^2\!\left(1
    - 2\,\frac{\vu{R}\cdot\vb{r}}{R}
    + \frac{r^2}{R^2}\right)
  \equiv R^2(1 - \epsilon)
\end{equation}
with $\epsilon = 2\,\vu{R}\cdot\vb{r}/R - r^2/R^2 = O(r/R)$. Then
\begin{equation}
\frac{1}{\lvert\vb{R} - \vb{r}\rvert^3}
  = \frac{1}{R^3}(1-\epsilon)^{-3/2}
  = \frac{1}{R^3}\!\left(1
    + \frac{3\,\vu{R}\cdot\vb{r}}{R}
    + O\!\left(\frac{r^2}{R^2}\right)\right)
\end{equation}

Multiply by $(\vb{R} - \vb{r})$ and keep terms through first order:
\begin{equation}
\frac{\vb{R} - \vb{r}}{\lvert\vb{R} - \vb{r}\rvert^3}
  = \frac{1}{R^3}\Big[
    \vb{R} + 3(\vu{R}\cdot\vb{r})\,\vu{R} - \vb{r}
  \Big]
  + O\!\left(\frac{r^2}{R^4}\right)
\end{equation}

Subtract $\vb{R}/R^3$:
\begin{equation}
\vb{f}(\vb{r})
  = \frac{\vb{R} - \vb{r}}{\lvert\vb{R} - \vb{r}\rvert^3}
  - \frac{\vb{R}}{R^3}
  = \frac{1}{R^3}\Big[
    3(\vu{R}\cdot\vb{r})\,\vu{R} - \vb{r}
  \Big]
  + O\!\left(\frac{r^2}{R^4}\right)
\end{equation}

The $\vb{R}/R^3$ terms cancel identically --- confirming that the
direct pull drops out --- and the leading surviving term is the tidal
quadrupole, Eq.~\eqref{eq:tidal}.\hfill$\square$

\medskip

We summarize the result of this section as a general theorem.

\begin{theorem}[Cancellation of direct gravitational pull]
\label{thm:cancel}
Let a test mass at position~$\vb{x}$ be subject to the gravitational
fields of the Earth~($M_E$) and an external body~($M$, at geocentric
distance $R \gg r$) together with a non-gravitational spring
force~$\vb{F}_{\emph{spring}}$. In the geocentric frame, the spring
force required to keep the test mass at rest is
\begin{equation*}
\frac{\vb{F}_{\emph{spring}}}{m}
  = \frac{GM_E}{r^2}\,\hat{\vb{r}}
  - \vb{a}_{\emph{tidal}}
\end{equation*}
where $\vb{a}_{\emph{tidal}} = (GM/R^3)[3(\hat{\vb{R}}\cdot\vb{r})
\hat{\vb{R}} - \vb{r}] + O(r^2/R^4)$ is the tidal acceleration.
The direct pull $GM/R^2$ of the external body does not appear: it
cancels exactly against the free-fall acceleration of the laboratory.
This holds for any measurement axis~$\hat{\vb{n}}$, any number of
external bodies, and to all orders in $r/R$.
\end{theorem}

The remainder of this paper verifies that rotating-frame corrections, sensor orientation, and all other terrestrial effects leave this conclusion intact.

\section{Transformation to the Laboratory Frame}
\label{sec:lab}

Equation~\eqref{eq:relative} is written in the non-rotating geocentric
frame. The laboratory rotates with the Earth at angular
velocity~$\boldsymbol{\omega}$. This is a second source of
non-inertiality (see Sec.~\ref{sec:frames}).

\subsection{Rotating-frame equation of motion}

Let $\vb{r}'$ be the test mass position in the lab frame, related to
its geocentric position by
$\vb{r} = \mathcal{R}(t)\,\vb{r}'$
where $\mathcal{R}(t)$ is the time-dependent rotation matrix. For
uniform rotation about axis
$\hat{\vb{n}} = \boldsymbol{\omega}/\omega$ at angular
rate $\omega = \lvert\boldsymbol{\omega}\rvert$, it takes the
Rodrigues form
\begin{equation}
\mathcal{R}(t)
  = \vb{I}\cos\omega t
  + (1 - \cos\omega t)\,\hat{\vb{n}}\,\hat{\vb{n}}^T
  + \sin\omega t\;[\hat{\vb{n}}]_\times
\label{eq:rodrigues}
\end{equation}
where $[\hat{\vb{n}}]_\times$ is the skew-symmetric matrix whose
action on any vector~$\vb{v}$ is
$[\hat{\vb{n}}]_\times\vb{v} = \hat{\vb{n}} \times \vb{v}$
--- i.e.\ the matrix representation of the cross product. Its time
derivative satisfies
\begin{equation}
\dot{\mathcal{R}}(t)
  = [\boldsymbol{\omega}]_\times\;\mathcal{R}(t)
\label{eq:Rdot}
\end{equation}
so that
$\dot{\mathcal{R}}\,\vb{v}
  = \boldsymbol{\omega} \times (\mathcal{R}\,\vb{v})$
for every vector~$\vb{v}$. Each time derivative
of~$\mathcal{R}$ thus generates a cross product
with~$\boldsymbol{\omega}$.

Differentiating $\vb{r} = \mathcal{R}\,\vb{r}'$ twice using
Eq.~\eqref{eq:Rdot}:
\begin{equation}
\ddot{\vb{r}}
  = \ddot{\vb{r}}'
  + 2\,\boldsymbol{\omega} \times \dot{\vb{r}}'
  + \boldsymbol{\omega} \times (\boldsymbol{\omega} \times \vb{r}')
\label{eq:kinematic}
\end{equation}
(the Euler term $\dot{\boldsymbol{\omega}} \times \vb{r}'$ vanishes for
uniform rotation). Note that $\boldsymbol{\omega}$ carries no prime:
$\mathcal{R}(t)$ is a rotation \emph{about}~$\boldsymbol{\omega}$,
so $\mathcal{R}^{-1}\boldsymbol{\omega} = \boldsymbol{\omega}$ ---
the rotation vector is invariant under the rotation it generates.
Substituting into Eq.~\eqref{eq:relative} and
expressing all vectors in the lab basis:
\begin{equation}
\ddot{\vb{r}}'
  = \underbrace{-\frac{G M_E}{\lvert\vb{r}'\rvert^3}\,\vb{r}'
    }_{\text{Earth's gravity}}
  \;+\; \underbrace{\vb{a}_{\text{tidal}}'
    }_{\text{tidal}}
  \;\underbrace{-\; \boldsymbol{\omega}
    \times (\boldsymbol{\omega} \times \vb{r}')
    }_{\text{centrifugal}}
  \;\underbrace{-\; 2\,\boldsymbol{\omega}
    \times \dot{\vb{r}}'
    }_{\text{Coriolis}}
  \;+\; \frac{\vb{F}_{\text{spring}}'}{m}
\label{eq:lab}
\end{equation}

The direct solar and lunar pulls do \textbf{not} reappear --- they
canceled in Eq.~\eqref{eq:relative} before the frame change, and
rotating the coordinate basis cannot restore a term that is already
zero.

\subsection{What the sensor reads}

The sensor housing is held in place by the laboratory floor, which
provides whatever normal force~$\vb{N}$ is required to prevent
acceleration. This constraint force acts on the housing, not on the
test mass, and therefore does not appear in Eq.~\eqref{eq:lab}. Its
effect enters through the spring: the housing end of the spring
follows the Earth's motion, so the spring
force~$\vb{F}_{\text{spring}}'$ adjusts to keep the test mass
co-moving.

The spring force is thus entirely \emph{caused by} the constraint.
The floor holds the housing fixed; the spring transmits that
constraint to the test mass. This is the equivalence principle in
action: if the floor were removed and the entire apparatus fell
freely, housing and test mass would share the same gravitational
acceleration. The spring would relax and
$\vb{F}_{\text{spring}}' \to \vb{0}$ --- precisely because no
constraint force distinguishes the housing from the test mass.
A spring-based sensor can only measure accelerations that
\emph{differ} between its two ends, i.e.\ tidal effects and
non-gravitational forces.

The test mass is thus at rest in the lab:
$\dot{\vb{r}}' = \ddot{\vb{r}}' = \vb{0}$. The Coriolis term
vanishes. Setting the left-hand side of Eq.~\eqref{eq:lab} to zero
determines the spring force --- the quantity the sensor reads:
\begin{equation}
\frac{\vb{F}_{\text{spring}}'}{m}
  = \frac{G M_E}{\lvert\vb{r}'\rvert^3}\,\vb{r}'
  + \boldsymbol{\omega} \times (\boldsymbol{\omega} \times \vb{r}')
  - \vb{a}_{\text{tidal}}'
\label{eq:spring}
\end{equation}

The sensor projects this onto its measurement axis~$\vu{n}$:
\begin{equation}
g_{\text{measured}}
  = \frac{\vb{F}_{\text{spring}}'}{m} \cdot \vu{n}
\end{equation}

The three contributions are:
\begin{itemize}
\item $G M_E\,\vb{r}'/\lvert\vb{r}'\rvert^3$: Earth's gravity (the
  dominant, static term).
\item $\boldsymbol{\omega}\times(\boldsymbol{\omega}\times\vb{r}')
  = -\omega^2\vb{r}'_\perp$: the centrifugal reduction (objects weigh
  less at the equator by ${\sim}\,0.3\%$).
\item $-\vb{a}_{\text{tidal}}'$: the tidal perturbation from Sun and
  Moon, of order $10^{-7}$~m/s$^2$.
\end{itemize}

\noindent
The direct pull $GM_S/R^2$ and $GM_M/D^2$ appear nowhere. This holds
for any~$\vu{n}$.

\subsection{Size of each term}

Table~\ref{tab:terms} lists every acceleration that enters
Eq.~\eqref{eq:spring}, together with the two direct pulls that
cancel before reaching it.

\begin{table}[h]
\caption{Magnitude of each acceleration term at Earth's surface,
ordered by size. The grayed-out rows --- the direct gravitational
pulls --- cancel exactly (Sec.~\ref{sec:proof}) and do not appear in
the sensor reading [Eq.~\eqref{eq:spring}]. The last column gives
the ratio of each term to the solar tidal residual
($5.0 \times 10^{-7}$~m/s$^2$); for the canceled rows it is the
ratio of direct pull to corresponding tidal residual ($11{,}700\times$
for the Sun, $30\times$ for the Moon).}
\label{tab:terms}
\begin{ruledtabular}
\begin{tabular}{llrrr}
\textsc{Term} & \textsc{Expression}
  & \textsc{Magnitude} (m/s$^2$)
  & \textsc{Daily variation} (m/s$^2$)
  & \textsc{Ratio} \\
\hline
Earth's gravity
  & $GM_E/r_\oplus^2$
  & $9.82$
  & static
  & $2 \times 10^7$ \\
Centrifugal ($\lambda\!=\!48.4°$)
  & $\omega^2 r_\oplus \cos^2\!\lambda$
  & $3.4 \times 10^{-2}\cos^2\!\lambda$
  & static
  & $7 \times 10^{4}\cos^2\!\lambda$ \\
\textcolor{gray}{Solar direct pull (canceled)}
  & \textcolor{gray}{$GM_S / R^2$}
  & \textcolor{gray}{$5.93 \times 10^{-3}$}
  & \textcolor{gray}{---}
  & \textcolor{gray}{$11{,}700\times$} \\
\textcolor{gray}{Lunar direct pull (canceled)}
  & \textcolor{gray}{$GM_M / D^2$}
  & \textcolor{gray}{$3.32 \times 10^{-5}$}
  & \textcolor{gray}{---}
  & \textcolor{gray}{$30\times$} \\
Lunar tide
  & $GM_M\, r_\oplus / D^3$
  & $1.1 \times 10^{-6}$
  & ${\lesssim}\; 2 \times 10^{-6}$
  & --- \\
Solar tide
  & $GM_S\, r_\oplus / R^3$
  & $5.0 \times 10^{-7}$
  & ${\lesssim}\; 8 \times 10^{-7}$
  & --- \\
\end{tabular}
\end{ruledtabular}
\end{table}

The hierarchy is striking. Earth's gravity dominates at
${\sim}\,10$~m/s$^2$. The centrifugal correction is four orders of
magnitude smaller (${\sim}\,10^{-2}$~m/s$^2$), yet still measurable ---
it is the reason objects weigh less at the equator than at the poles.
The tidal accelerations are smaller still, at $10^{-6}$--$10^{-7}$~m/s$^2$,
but are routinely detected by superconducting gravimeters and satellite
missions such as GRACE.

The canceled direct pulls occupy a revealing intermediate scale.
The Sun's direct pull ($5.93 \times 10^{-3}$~m/s$^2$) is
$180\times$ stronger than the Moon's ($3.32 \times 10^{-5}$~m/s$^2$),
yet its tidal effect is $2.2\times$ \emph{weaker}
($1/R^3$ vs.\ $1/R^2$ scaling). This inversion --- the Sun pulls
harder but tidally disturbs less --- is a direct signature of the
cancellation. If direct pulls were observable, the Sun would dominate
the signal by two orders of magnitude. Instead, the Moon dominates
the tidal signal, exactly as observed.

The sensor reads a superposition of the four non-grayed terms in
Table~\ref{tab:terms}. The direct pulls have been subtracted out by
the physics of the measurement: the lab accelerates with the Earth,
and the spring cannot see it.


\section{Why Sensor Orientation Is Irrelevant}
\label{sec:orientation}

The cancellation in Eq.~\eqref{eq:cancel} is \textbf{vectorial}: the
direct pull vanishes as a three-component vector, not as a particular
scalar projection. Projecting onto any measurement axis~$\vu{n}$:
\begin{equation}
\vu{n} \cdot \vb{f}(\vb{0}) = \vu{n} \cdot \vb{0} = 0
  \qquad\text{for all } \vu{n}
\end{equation}

Tilting the sensor changes~$\vu{n}$ but cannot make zero non-zero.
The spring force along any axis contains only the tidal terms from
Eq.~\eqref{eq:tidal}.

Physically: the sensor housing and the test mass both accelerate at
$GM_S/R^2$ toward the Sun. The spring connecting them measures only the
\emph{difference} between the accelerations of its two endpoints. A
uniform field produces no difference, regardless of the spring's
orientation.

\subsection{Sweeping the sensor orientation}

Parameterise the measurement axis by zenith angle~$\theta$ and
azimuth~$\varphi$:
\begin{equation}
\vu{n}(\theta,\varphi)
  = \sin\theta\cos\varphi\;\vu{E}
  + \sin\theta\sin\varphi\;\vu{N}
  + \cos\theta\;\vu{U}
\end{equation}
where $\vu{E}$, $\vu{N}$, $\vu{U}$ are the local East--North--Up
unit vectors defined relative to the \emph{geodetic} vertical.
Projecting Eq.~\eqref{eq:spring} onto $\vu{n}(\theta,\varphi)$
and writing each term explicitly:
\begin{align}
g(\theta,\varphi)
  &= \underbrace{\frac{GM_E}{r_\oplus^2}\cos\theta
       \;+\; \bigl[\boldsymbol{\omega}\times
       (\boldsymbol{\omega}\times\vb{r}')\bigr]
       \cdot\vu{n}}_{\text{gravity + centrifugal}}
  \;+\; \underbrace{(-2\boldsymbol{\omega}
       \times\dot{\vb{r}}')\cdot\vu{n}}_{=\,0\;\text{(static)}}
  \;-\; \vb{a}_{\text{tidal}}'\cdot\vu{n}
  \notag \\[6pt]
  &= g_0\cos\theta
  \;-\; a_E'\sin\theta\cos\varphi
  \;-\; a_N'\sin\theta\sin\varphi
  \;-\; a_U'\cos\theta
\label{eq:sweep}
\end{align}
In the first line, the gravitational and centrifugal projections
combine into a single $\cos\theta$ term because both point along
the geodetic vertical~$\vu{U}$ (see Appendix~\ref{app:coordinates}
for a detailed discussion of why the centrifugal term has no
azimuthal component in this frame), and the Coriolis term vanishes
for a stationary test mass ($\dot{\vb{r}}' = \vb{0}$). The result defines the effective
gravity $g_0 \equiv GM_E/r_\oplus^2
- \omega^2 r_\oplus\cos^2\!\lambda$ (where $\lambda$ is the geodetic
latitude), absorbing the centrifugal correction into the amplitude.

We now trace each contribution through this projection.

\textbf{Earth's gravity and centrifugal ($g_0\cos\theta$).}
The effective gravity
$\vb{g}_{\text{eff}} = g_0\,\vu{U}$ points purely along the
geodetic vertical by definition.
Its projection onto the sensor axis is $g_0\cos\theta$: it reads
the full $g_0 \approx 9.81$~m/s$^2$ when the sensor is vertical
($\theta = 0$), vanishes when horizontal ($\theta = 90°$), and
reverses when inverted ($\theta = 180°$). The centrifugal correction
modifies~$g_0$ by at most $0.3\%$ (latitude-dependent) but
introduces no separate angular signature --- it is entirely absorbed
into the amplitude of the $\cos\theta$ term. There is no
``rotational residual'' in the sweep.

\textbf{Solar and lunar tides ($a_E'$, $a_N'$, $a_U'$).}
The tidal accelerations from Eq.~\eqref{eq:tidal} decompose into
East, North, and Up components. The horizontal components $a_E'$
and $a_N'$ enter through $\sin\theta\cos\varphi$ and
$\sin\theta\sin\varphi$: they vanish for a vertical sensor and are
maximized for a horizontal one pointed in the appropriate direction.
The vertical component $a_U'$ enters as $\cos\theta$, superimposed
on the much larger $g_0\cos\theta$. Representative combined
amplitudes (from Sec.~\ref{sec:lab}): solar tidal
${\sim}\,5.0 \times 10^{-7}$~m/s$^2$, lunar tidal
${\sim}\,1.1 \times 10^{-6}$~m/s$^2$. These are the only
time-varying signals in the sweep.

\textbf{Direct solar pull (canceled).}
Were the direct pull $GM_S/R^2$ not canceled by Earth's free fall,
it would project as
$a_\odot\sin\theta\cos(\varphi - \varphi_\odot)$
with amplitude $a_\odot \approx 5.93 \times 10^{-3}$~m/s$^2$ ---
four orders of magnitude above the tidal terms. As shown in
Fig.~\ref{fig:sweep}, no such signal exists at any orientation.
An angular scan therefore confirms the cancellation directly: one
measures $g_0\cos\theta$ plus tidal corrections of order
$10^{-7}$~m/s$^2$, with no $10^{-3}$~m/s$^2$ sinusoidal component.

\begin{figure}[t]
\centering
\includegraphics[width=\columnwidth]{figures/orientation_sweep.pdf}
\caption{Residual acceleration $g - g_\mathrm{grav}\cos\theta$ as a function of
sensor orientation, where $g_\mathrm{grav}$ is the pure gravitational
acceleration (without centrifugal). Left column: sweep over zenith
angle~$\theta$ at fixed azimuth $\varphi = 0°$ (East). Right column:
sweep over azimuth~$\varphi$ at fixed $\theta = 45°$.
\textbf{Top row}~(a,\,b): $10^{-3}$~m/s$^2$ scale.
Solid blue: centrifugal correction
($\omega^2 r_\oplus \cos^2\!\lambda \approx 1.7 \times 10^{-2}$~m/s$^2$
at Ulm, $\lambda = 48.4°$);
red dashed: hypothetical Sun direct pull
($5.93 \times 10^{-3}$~m/s$^2$);
orange dashed: hypothetical Moon direct pull
($3.3 \times 10^{-5}$~m/s$^2$);
blue solid: combined tidal residual (indistinguishable from zero at this
scale). The centrifugal term is the largest non-gravitational contribution
and exceeds even the (canceled) Sun direct pull.
\textbf{Bottom row}~(c,\,d): $10^{-7}$~m/s$^2$ scale, zoomed to the
actual tidal signals. Teal: solar tidal; magenta: lunar tidal;
blue: combined. The centrifugal ($10^5\times$), canceled direct Sun
($10^4\times$), and Moon ($30\times$) pulls are all off this scale.
No orientation reveals the claimed $10^{-3}$~m/s$^2$ signal.}
\label{fig:sweep}
\end{figure}


\section{What IS Observable}
\label{sec:observable}

Given that the direct pulls cancel, what \emph{does} a tilted
sensor actually measure? From Eq.~\eqref{eq:sweep}, the sensor
reading decomposes into three physically distinct contributions.
Each responds differently to changes in orientation and time.

\subsection{Static baseline: gravity and centrifugal}

The dominant term $g_0\cos\theta$ is the projection of the
effective gravity onto the measurement axis. It is entirely static
--- independent of time, the positions of Sun and Moon, or
anything else in the sky. Tilting the sensor from vertical
($\theta = 0$) to horizontal ($\theta = 90°$) sweeps this term
from $g_0 \approx 9.81$~m/s$^2$ through zero.

Embedded within $g_0$ is the centrifugal correction, which at Ulm
amounts to $\omega^2 r_\oplus\cos^2\!\lambda \approx
1.5 \times 10^{-2}$~m/s$^2$. It projects as
$1.5 \times 10^{-2}\cos\theta$~m/s$^2$ and is visible in
Fig.~\ref{fig:sweep} (panels~a,\,b) as the solid blue curve.
This is the largest non-gravitational contribution the sensor
can detect --- it exceeds even the (canceled) solar direct pull
by a factor of~$2.5$. Unlike the tidal terms, however, it carries
no time dependence: it is a fixed geometric projection that shifts
the baseline reading as the sensor is tilted.

\subsection{Time-varying signals: tidal components}

The only orientation-dependent terms that vary in time are the
tidal projections $a_E'\sin\theta\cos\varphi$,
$a_N'\sin\theta\sin\varphi$, and $a_U'\cos\theta$. These
encode the three components of the tidal acceleration vector
in the geodetic frame:

\begin{itemize}
\item A vertical sensor ($\theta = 0$) measures $a_U'$, the
  vertical tidal component (${\sim}\,5.0 \times 10^{-7}$~m/s$^2$,
  semi-diurnal with ${\sim}\,12$~h period).
\item A horizontal sensor ($\theta = 90°$) pointing East
  ($\varphi = 0$) measures $a_E'$; pointing North
  ($\varphi = 90°$) it measures $a_N'$. These horizontal
  components are diurnal (${\sim}\,24$~h period).
\item Three non-coplanar sensors reconstruct the full tidal
  acceleration vector and, in principle, the tidal tensor.
\end{itemize}

\noindent
The lunar tidal signal (${\sim}\,1.1 \times 10^{-6}$~m/s$^2$) is
roughly twice the solar (${\sim}\,5.0 \times 10^{-7}$~m/s$^2$),
consistent with the $1/R^3$ scaling that favors the nearby Moon
over the distant Sun. Superconducting gravimeters routinely resolve
these signals; they are the standard observable in terrestrial
tidal gravimetry.

\subsection{What is absent}

Conspicuously absent from Eq.~\eqref{eq:sweep} are two terms
that would dominate the reading if direct gravitational pulls
were not canceled:

\begin{itemize}
\item A \textbf{solar} term of order
  $5.93 \times 10^{-3}$~m/s$^2$, projecting as
  $a_\odot\sin\theta\cos(\varphi - \varphi_\odot)$ with a
  diurnal period (${\sim}\,24$~h). This would exceed the
  actual tidal signals by four orders of magnitude.
\item A \textbf{lunar} term of order
  $3.32 \times 10^{-5}$~m/s$^2$, projecting as
  $a_M\sin\theta\cos(\varphi - \varphi_M)$
  with a period of ${\sim}\,24.8$~h. Smaller than the solar
  term by a factor of~$180$, but still $30\times$ larger than
  the lunar tidal signal it would accompany.
\end{itemize}

\noindent
Figure~\ref{fig:sweep} confirms their absence directly: the top
panels show both hypothetical direct pulls as dashed curves that
dwarf the actual tidal residual (blue, indistinguishable from
zero at this scale). The bottom panels zoom into the
$10^{-7}$~m/s$^2$ scale where the real tidal signals live ---
orders of magnitude below either claimed effect.

All orientation-dependent signals the sensor can detect ---
centrifugal and tidal --- are real and well understood.
None is a direct gravitational pull.


\section{The Equivalence Principle}
\label{sec:ep}

The cancellation derived in Sec.~\ref{sec:proof} is not a
calculational coincidence. It is the equivalence principle.

Einstein's key insight was that a uniform gravitational field is
locally indistinguishable from an accelerating reference frame.
No experiment confined to a sufficiently small laboratory can
determine whether the laboratory is at rest in a gravitational
field or accelerating through empty space. This means that a
freely falling observer cannot detect the uniform component of any
gravitational field --- only its \emph{gradient} (the tidal part)
is locally measurable.

The Earth, together with the sensor, the laboratory, and the
observer, is in free fall in the Sun's gravitational field. The
entire system accelerates at $GM_S/R^2$ toward the Sun. A
spring-based sensor measures the force difference between its two
endpoints. Because both endpoints share the same free-fall
acceleration, the spring is blind to the Sun's uniform pull. Only
the \emph{non-uniformity} of the field across the sensor ---
the tidal term, of order $GM_S\,r_\oplus/R^3$ --- can produce a
differential signal.

In the language of general relativity, the Earth follows a geodesic
in the Sun's spacetime. Along a geodesic, no proper acceleration
is felt --- the ``force of gravity'' is replaced by the geometry
of spacetime. What remains observable is geodesic deviation: nearby
geodesics converge or diverge due to spacetime curvature. This is
the tidal acceleration of Eq.~\eqref{eq:tidal}, encoded in the
Riemann tensor. The direct pull $GM_S/R^2$ has no counterpart in
this description; it is an artifact of the Newtonian decomposition
into ``gravitational force'' and ``inertial motion,'' a
decomposition that the equivalence principle declares unphysical.

This is why no adjustment of sensor orientation, no change of
measurement axis, and no increase in sensitivity can reveal a
$5.93 \times 10^{-3}$~m/s$^2$ signal from the Sun. The signal does
not exist at the sensor because the equivalence principle forbids
it. What the sensor can measure --- and what is measured routinely
--- are tidal effects at $10^{-7}$--$10^{-6}$~m/s$^2$, four orders
of magnitude below the direct pull. The proof in
Sec.~\ref{sec:proof} is the Newtonian statement of this principle;
the equivalence principle elevates it from a property of gravity to
a law of nature.

\appendix

\section{Three-point formulation: housing as a separate body}
\label{app:three-point}

The main text treats the sensor housing as coincident with Earth's
center. Here we promote the housing to a separate coordinate and
show that no new physics arises.

\subsection{Setup}

A real sensor does not sit at Earth's center; it is bolted to the
crust at some location on the surface. One might therefore ask
whether the cancellation of Sec.~\ref{sec:proof} relies on the
idealisation of a test mass at $\vb{R}_E$. To address this, we
introduce three distinct positions in the inertial frame:
\begin{itemize}
\item Earth's center: $\vb{R}_E(t)$, whose motion is governed by
  the combined gravitational attraction of all external bodies.
\item Housing (spring attachment point):
  $\vb{R}_H(t) = \vb{R}_E(t) + \vb{d}(t)$,
  where $\lvert\vb{d}\rvert = r_\oplus$ and $\vb{d}$ is fixed in
  the rotating frame. The housing is rigidly attached to the
  crust (Born rigidity): it co-rotates with the Earth and is held
  in place by internal stresses rather than orbiting freely.
\item Test mass: $\vb{x}(t)$, connected to the housing by a spring
  of rest length~$\ell$.
\end{itemize}

\noindent
The key dynamical variable is the \emph{spring extension},
\begin{equation}
\vb{s} \;\equiv\; \vb{x} - \vb{R}_H,
\label{eq:spring-ext}
\end{equation}
with $\lvert\vb{s}\rvert \sim \text{cm}$. This is the vector that the
spring force acts along, and it is the quantity that the sensor
ultimately transduces into a reading. Everything the sensor can
measure must be encoded in~$\vb{s}$ and its time derivatives.

\subsection{Equations of motion}

We now write Newton's second law for both the test mass and the
housing separately. This is the crucial difference from the main
text, where both were lumped at Earth's center. We denote the
external body's mass by~$M$ and its inertial position
by~$\vb{R}_B(t)$.

\textbf{Test mass.} The test mass is subject to gravitational
attraction from the Earth and the external body, plus the spring force.
Its equation of motion is identical to Eq.~\eqref{eq:test}:
\begin{equation}
m\,\ddot{\vb{x}}
  = -\frac{G M_E\, m}{\lvert\vb{x} - \vb{R}_E\rvert^3}
    (\vb{x} - \vb{R}_E)
  - \frac{G M\, m}{\lvert\vb{x} - \vb{R}_B\rvert^3}
    (\vb{x} - \vb{R}_B)
  + \vb{F}_{\text{spring}}
\label{eq:app-test}
\end{equation}
(the lunar term is omitted for brevity; it enters identically).

\textbf{Housing.} Unlike the test mass, the housing is not free:
it is a rigid extension of the Earth's crust. It feels gravity from
both the Earth and the external body, the reaction force of the spring
(Newton's third law), and a constraint (normal) force~$\vb{N}$
exerted by the crust that prevents it from falling:
\begin{equation}
m_H\,\ddot{\vb{R}}_H
  = -\frac{G M_E\, m_H}{\lvert\vb{d}\rvert^3}\,\vb{d}
  - \frac{G M\, m_H}{\lvert\vb{R}_H - \vb{R}_B\rvert^3}
    (\vb{R}_H - \vb{R}_B)
  + \vb{N}
  - \vb{F}_{\text{spring}}
\label{eq:app-housing}
\end{equation}
The constraint force~$\vb{N}$ is whatever is required to enforce
$\vb{d} = \text{const}$ in the rotating frame --- it is the
mechanical reaction of the ground on the sensor mount. Note that
$\vb{N}$ encodes the full weight of the housing in Earth's gravity
field, including the tidal gradient across $r_\oplus$. This will
become important in Sec.~\ref{app:why-main}.

\subsection{Subtraction: equation for the spring extension}

The sensor measures the spring extension~$\vb{s}$, not the absolute
position of either endpoint. To obtain the equation governing~$\vb{s}$,
we subtract the housing equation~\eqref{eq:app-housing} (divided
by~$m_H$) from the test-mass equation~\eqref{eq:app-test} (divided
by~$m$). Using $\ddot{\vb{s}} = \ddot{\vb{x}} - \ddot{\vb{R}}_H$ and
writing $\vb{R} = \vb{R}_B - \vb{R}_E$:
\begin{align}
m\,\ddot{\vb{s}}
  &= -\frac{G M_E\, m}{\lvert\vb{d} + \vb{s}\rvert^3}
      (\vb{d} + \vb{s})
    + \frac{G M_E\, m}{d^3}\,\vb{d}
  \notag \\
  &\quad + G M\, m \left[
    \frac{\vb{R} - \vb{d} - \vb{s}}
         {\lvert\vb{R} - \vb{d} - \vb{s}\rvert^3}
    - \frac{\vb{R} - \vb{d}}
         {\lvert\vb{R} - \vb{d}\rvert^3}
  \right]
  \notag \\
  &\quad + \left(1 + \frac{m}{m_H}\right)\vb{F}_{\text{spring}}
    + \frac{m}{m_H}\,\vb{N}
\label{eq:app-rel}
\end{align}

\subsection{The cancellation}

The key question is whether the external body's direct gravitational pull
$GM/R'^2$ survives in the equation for $\vb{s}$. Examine the
external-body term in Eq.~\eqref{eq:app-rel}. It has the same
structure as Eq.~\eqref{eq:relative} of the main text:
\begin{equation}
\vb{g}(\vb{s})
  = \frac{\vb{R}' - \vb{s}}{\lvert\vb{R}' - \vb{s}\rvert^3}
  - \frac{\vb{R}'}{R'^3}
\label{eq:app-tidal}
\end{equation}
where $\vb{R}' \equiv \vb{R} - \vb{d}$ is the external body's position relative
to the \emph{housing} (not Earth's center). Evaluating at
$\vb{s} = \vb{0}$ (i.e.\ when the test mass sits exactly at the
housing):
\begin{equation}
\vb{g}(\vb{0}) = \vb{0}
  \qquad\text{(exactly)}
\end{equation}

This is the same cancellation as in Sec.~\ref{sec:proof}, but now
its physical meaning is sharper: the direct pull $GM/R'^2$ on the
test mass cancels against the direct pull on the housing --- not
against Earth's center, but against the other end of the spring.
Both spring endpoints are pulled toward the external body by the same
acceleration (to leading order in $s/R'$), so the spring itself
cannot detect it. The cancellation is local: it occurs at the
sensor, between its two endpoints.

\subsection{What survives: tidal across the spring}

With the direct pull gone, what remains? Expanding
Eq.~\eqref{eq:app-tidal} to leading order in $s/R'$ (with
$R' \approx R$ since $d \ll R$):
\begin{equation}
\vb{a}_{\text{tidal}}^{(\text{spring})}
  = \frac{GM}{R'^3}\Big[
    3\,(\hat{\vb{R}}'\cdot\vb{s})\,\hat{\vb{R}}' - \vb{s}
  \Big]
  + O\!\left(\frac{s^2}{R'^4}\right)
\label{eq:app-tidal-spring}
\end{equation}

This is the tidal acceleration across the \emph{spring baseline}
$\vb{s}$, not across $r_\oplus$. The tidal field is a gradient:
it stretches space differentially, and the effect scales linearly
with the baseline length. For a spring of length
$\ell \sim 1$~cm, the ratio to the Earth-radius tidal signal is:
\begin{equation}
\frac{a_{\text{tidal}}^{(\text{spring})}}{a_{\text{tidal}}^{(r_\oplus)}}
  \;\sim\; \frac{\ell}{r_\oplus}
  \;\sim\; \frac{10^{-2}}{6.4 \times 10^6}
  \;\sim\; 1.6 \times 10^{-9}
\end{equation}

The tidal acceleration across the spring is therefore
${\sim}\,5 \times 10^{-7} \times 1.6 \times 10^{-9}
\approx 10^{-15}$~m/s$^2$ --- unmeasurably small by any current
technology. This is the \emph{only} external-body signal intrinsic to the
spring in the three-point formulation. If a gravimeter observes
tidal signals at $10^{-7}$~m/s$^2$, the signal must originate
elsewhere. As the next subsection shows, it is transmitted
through the constraint force~$\vb{N}$.

\subsection{Why the main text uses Earth's center}
\label{app:why-main}

The reason gravimeters observe tidal signals at
$10^{-7}$~m/s$^2$ and not at $10^{-15}$~m/s$^2$ is that
the \emph{constraint force}~$\vb{N}$ in
Eq.~\eqref{eq:app-rel} transmits Earth's gravity gradient
across the full radius~$r_\oplus$ to the housing. The spring
then measures the difference between Earth's gravity at
$\vb{d} + \vb{s}$ and at $\vb{d}$, which includes the
$r_\oplus$-scale tidal contribution already encoded in~$\vb{N}$.

Absorbing $\vb{N}$ and the Earth gravity terms into the
effective gravity $g_0$ at the housing reproduces
Eq.~\eqref{eq:spring} of the main text. The three-point
formulation thus reduces to the two-point formulation: no
information is lost and no new signal appears. The direct
pull $GM/R^2$ cancels in both formulations, for
the same reason --- the housing and the test mass share
the same free-fall acceleration.

The rotating-frame corrections of Sec.~\ref{sec:lab} carry over
unchanged. Transforming to the co-rotating frame introduces
centrifugal and Coriolis terms for both the test mass and the
housing. In the equation for $\vb{s}$, these fictitious forces
appear as the \emph{difference} between their values at
$\vb{R}_H + \vb{s}$ and at $\vb{R}_H$. Since both vary smoothly
on the scale of $r_\oplus$, their difference across the spring
baseline $\lvert\vb{s}\rvert \sim \ell$ is suppressed by the same
factor $\ell/r_\oplus \sim 10^{-9}$ as the tidal terms --- negligible.
The Coriolis force $-2m\,\boldsymbol{\omega}\times\dot{\vb{s}}$
vanishes identically for a static test mass ($\dot{\vb{s}} = \vb{0}$),
just as in the main text. The full centrifugal correction at the
housing location is, once again, absorbed into~$\vb{N}$ and hence
into~$g_0$. No new rotational effect emerges in the three-point
formulation.


\section{Coordinate systems and the centrifugal projection}
\label{app:coordinates}

Section~\ref{sec:orientation} states that the centrifugal
contribution to the sensor reading has no azimuthal dependence:
it enters purely as a correction to~$g_0\cos\theta$, with no
$\sin\theta\sin\varphi$ or $\sin\theta\cos\varphi$ term. This
is not obvious \emph{a priori}~--- the centrifugal acceleration
has a horizontal component that could, in principle, project
differently for different azimuths. The resolution lies in the
choice of coordinate system: the ENU frame used in the main text
is \emph{geodetic}, not geocentric, and the geodetic vertical
absorbs the horizontal centrifugal component by definition. This
appendix makes the argument explicit.

\subsection{The inertial (geocentric) frame}

We begin in a non-rotating frame centered on the Earth, using
spherical coordinates aligned with the rotation axis. At a point~P
on the surface at geocentric colatitude~$\Theta$ (i.e.\
geocentric latitude $\lambda' = 90° - \Theta$), define a
local geocentric triad:
\begin{align}
\hat{\vb{U}}' &= \hat{\vb{r}} &&\text{(radially outward from
  Earth's center)} \notag \\
\hat{\vb{N}}' &= -\hat{\boldsymbol{\Theta}}
  &&\text{(northward along the meridian)} \notag \\
\hat{\vb{E}}' &= \hat{\boldsymbol{\varphi}}_{\!\text{long}}
  &&\text{(eastward)}
\label{eq:geocentric-triad}
\end{align}
In this frame, the gravitational acceleration of a spherically
symmetric Earth is purely radial:
\begin{equation}
\vb{g}_{\text{grav}} = -g_{\text{grav}}\;\hat{\vb{U}}'
\label{eq:ggrav}
\end{equation}
with $g_{\text{grav}} = GM_E/r_\oplus^2 \approx 9.82$~m/s$^2$.
There is no horizontal gravitational component: $\vb{g}_{\text{grav}}$
has no $\hat{\vb{N}}'$ or $\hat{\vb{E}}'$ projection.

The centrifugal acceleration at~P points radially outward from
the rotation axis (not from Earth's center). The rotation axis
lies in the local meridional plane, so the centrifugal vector
has no eastward component but decomposes into radial and
meridional parts:
\begin{equation}
\vb{a}_{\text{cf}}
  = \omega^2 r_\oplus\cos^2\!\lambda'\;\hat{\vb{U}}'
  \;-\; \omega^2 r_\oplus\cos\lambda'\sin\lambda'\;\hat{\vb{N}}'
\label{eq:acf-geocentric}
\end{equation}
The first term ($\hat{\vb{U}}'$) reduces the apparent weight;
the second ($\hat{\vb{N}}'$) pushes equatorward. At
$\lambda' = 48.4°$ (Ulm), the horizontal component has magnitude
$\omega^2 r_\oplus\cos\lambda'\sin\lambda' \approx 0.017$~m/s$^2$
--- small compared to $g_{\text{grav}}$ but three times larger
than the Sun's direct pull $a_\odot \approx 5.93 \times 10^{-3}$~m/s$^2$.

\textbf{Effective gravity in the geocentric frame.}
Combining Eqs.~\eqref{eq:ggrav} and~\eqref{eq:acf-geocentric}:
\begin{equation}
\vb{g}_{\text{eff}}
  = -(g_{\text{grav}}
      - \omega^2 r_\oplus\cos^2\!\lambda')\;\hat{\vb{U}}'
  \;-\; \omega^2 r_\oplus\cos\lambda'\sin\lambda'\;\hat{\vb{N}}'
\label{eq:geff-geocentric}
\end{equation}

This vector does \emph{not} point along $\hat{\vb{U}}'$: it is
tilted toward the equator. If we were to project
$\vb{g}_{\text{eff}}$ onto a sensor axis parameterised in the
geocentric triad,
$\hat{\vb{n}}'(\theta,\varphi)
= \sin\theta\cos\varphi\;\hat{\vb{E}}'
+ \sin\theta\sin\varphi\;\hat{\vb{N}}'
+ \cos\theta\;\hat{\vb{U}}'$,
we would find:
\begin{equation}
\vb{g}_{\text{eff}}\cdot\hat{\vb{n}}'
  = -(g_{\text{grav}}
      - \omega^2 r_\oplus\cos^2\!\lambda')\,\cos\theta
  \;-\; \omega^2 r_\oplus\cos\lambda'\sin\lambda'
        \,\sin\theta\sin\varphi
\label{eq:sweep-geocentric}
\end{equation}

The $\sin\theta\sin\varphi$ term is the azimuthal centrifugal
projection. It is largest when the sensor points North
($\varphi = 90°$, $\theta = 90°$) and vanishes when pointing
East ($\varphi = 0°$) or vertically ($\theta = 0$). Its amplitude
($\approx 0.017$~m/s$^2$) would be clearly visible in any
orientation sweep if the geocentric frame were used.

\subsection{The rotating (geodetic) frame}

A laboratory on Earth's surface does not use geocentric
coordinates. The plumb line defines the local vertical, and
instruments are levelled relative to it. The plumb line hangs
along the direction of $\vb{g}_{\text{eff}}$, not along
$\hat{\vb{r}}$. This motivates the geodetic triad:
\begin{align}
\hat{\vb{U}} &= -\frac{\vb{g}_{\text{eff}}}
  {\lvert\vb{g}_{\text{eff}}\rvert}
  &&\text{(along the plumb line, upward)} \notag \\
\hat{\vb{N}} &\perp \hat{\vb{U}},\;\text{in the meridional plane}
  &&\text{(geodetic northward)} \notag \\
\hat{\vb{E}} &= \hat{\vb{E}}'
  &&\text{(eastward, unchanged)}
\label{eq:geodetic-triad}
\end{align}

By construction, $\vb{g}_{\text{eff}}$ is purely along
$-\hat{\vb{U}}$:
\begin{equation}
\vb{g}_{\text{eff}} = -g_0\;\hat{\vb{U}}
\label{eq:geff-geodetic}
\end{equation}
where $g_0 = \lvert\vb{g}_{\text{eff}}\rvert$. The horizontal
centrifugal component has been absorbed into the \emph{direction}
of $\hat{\vb{U}}$ (tilted by the angle~$\alpha$ from the
geocentric radial) and into the \emph{magnitude}~$g_0$.

The tilt angle between the geodetic and geocentric verticals is:
\begin{equation}
\alpha \approx
  \frac{\omega^2 r_\oplus\cos\lambda'\sin\lambda'}{g_{\text{grav}}}
  \approx 0.10°
  \quad\text{at } \lambda' = 45°
\label{eq:vertical-deflection}
\end{equation}

This is the well-known \emph{deflection of the vertical} due to
Earth's rotation.

\textbf{Sensor projection in the geodetic frame.}
Parameterising the sensor axis in the geodetic triad,
$\hat{\vb{n}}(\theta,\varphi)
= \sin\theta\cos\varphi\;\hat{\vb{E}}
+ \sin\theta\sin\varphi\;\hat{\vb{N}}
+ \cos\theta\;\hat{\vb{U}}$,
the projection is simply:
\begin{equation}
\vb{g}_{\text{eff}}\cdot\hat{\vb{n}}
  = -g_0\,\cos\theta
\label{eq:sweep-geodetic}
\end{equation}

No azimuthal dependence survives. The centrifugal contribution is
entirely contained in~$g_0$ and appears only through $\cos\theta$.

\subsection{Reconciling the two frames}

Equations~\eqref{eq:sweep-geocentric} and~\eqref{eq:sweep-geodetic}
describe the same physics in different coordinates. They are
related by a rotation of the local triad by the angle~$\alpha$
about the East axis:
\begin{equation}
\begin{pmatrix} \hat{\vb{U}} \\ \hat{\vb{N}} \end{pmatrix}
= \begin{pmatrix}
  \cos\alpha & \sin\alpha \\
  -\sin\alpha & \cos\alpha
\end{pmatrix}
\begin{pmatrix} \hat{\vb{U}}' \\ \hat{\vb{N}}' \end{pmatrix}
\end{equation}

The angles $\theta$ and $\varphi$ defined relative to the geodetic
triad differ from those defined relative to the geocentric triad
by corrections of order~$\alpha \sim 10^{-3}$~rad ($\sim 0.1°$).
In the geodetic frame, the horizontal centrifugal component
vanishes identically, not approximately: it is absorbed exactly
into the definition of the vertical. In the geocentric frame, the
same information appears as an explicit $\sin\theta\sin\varphi$
term. The physics is identical; only the bookkeeping differs.

\subsection{Why the geodetic frame is natural}

Every laboratory-based sensor defines its orientation relative to
the plumb line. A tiltmeter measures deviation from the geodetic
vertical; a gravimeter is levelled to the plumb line; an
accelerometer's ``zero'' is set by the local~$g_0$. The geodetic
ENU frame is therefore the natural coordinate system for the
orientation sweep of Sec.~\ref{sec:orientation}.

In this frame, the centrifugal acceleration produces no azimuthal
signature. It is absorbed completely into the magnitude~$g_0$ that
multiplies $\cos\theta$. The only azimuth-dependent terms in the
sensor reading, Eq.~\eqref{eq:sweep}, are the tidal
components~$a_E'$ and~$a_N'$, at the $10^{-7}$~m/s$^2$ level.
The centrifugal horizontal component ($\approx 0.017$~m/s$^2$)
is real, but it is not a separate signal: it defines the direction
the sensor calls ``up.''


\section{Implications for atomic transitions and mass defect}
\label{app:mass-defect}

The cancellation theorem of Sec.~\ref{sec:proof} was derived for a
classical spring sensor. Here we show that the same cancellation
extends to quantum systems --- in particular to atomic transitions,
where the mass--energy equivalence $E = mc^2$ ties internal energy
directly to gravitational coupling.

\subsection{Mass defect, recoil, and gravitational coupling}

An atomic transition changes both the internal state and the
external momentum of the atom. Consider an atom initially at rest
in the lab frame that absorbs a photon of wave vector~$\vb{k}$.
After the transition:

\textbf{Internal state.}
The atom's rest mass increases by the mass defect
\begin{equation}
\Delta m = \frac{E_e - E_g}{c^2}
         = \frac{h\nu_0}{c^2}
\label{eq:mass-defect}
\end{equation}
where $\nu_0$ is the transition frequency.

\textbf{External state.}
The atom acquires a recoil momentum $\Delta\vb{p} = \hbar\vb{k}$,
producing a recoil velocity
\begin{equation}
\vb{v}_r = \frac{\hbar\vb{k}}{m_e}
\label{eq:recoil}
\end{equation}
where $m_e = m_g + h\nu_0/c^2$ is the excited-state mass.
The recoil velocity is state-dependent through~$m_e$: compared
to the ground-state value $\hbar k/m_g$, it differs by the
fractional amount $h\nu_0/(m_g c^2) \sim 10^{-10}$ for typical
optical transitions.

Both changes couple to gravity. By the equivalence principle, the
gravitational force on an atom in state~$n$ with total mass--energy
$m_n = m_0 + E_n/c^2$ is
\begin{equation}
\vb{F}_{\text{grav}}^{(n)}
  = -m_n\,\nabla\Phi
  = -\left(m_0 + \frac{E_n}{c^2}\right)\nabla\Phi
\label{eq:grav-internal}
\end{equation}
where $\Phi$ is the gravitational potential. The internal energy
$E_n$ enters the equation of motion on exactly the same
footing as the rest mass~$m_0$. Meanwhile, after the recoil, the
atom follows a trajectory governed by the same gravitational
acceleration~$-\nabla\Phi$ regardless of its internal state ---
only the initial velocity~$\vb{v}_r$ differs. Both the mass defect
(what the atom weighs) and the recoil (where the atom goes) are
subject to the same gravitational field.

\subsection{Cancellation for internal and external degrees of freedom}

Consider an atom at position~$\vb{x}$ in the field of the Earth
and a distant external body of mass~$M$ at position~$\vb{R}_B$. By Eq.~\eqref{eq:grav-internal}, the gravitational
acceleration is
\begin{equation}
\vb{a}_{\text{grav}}
  = -\nabla\Phi
  = -\frac{GM_E}{r^2}\,\hat{\vb{r}}
    - \frac{GM}{\lvert\vb{x} - \vb{R}_B\rvert^3}\,
      (\vb{x} - \vb{R}_B)
    + \cdots
\end{equation}
Crucially, this acceleration is \emph{independent of} $m_n$: the
factor $m_n$ in the gravitational force~\eqref{eq:grav-internal}
cancels against the $m_n$ in Newton's second law, $\vb{F} = m_n\,\vb{a}$.
This is the equivalence principle at work. An excited atom and a
ground-state atom fall with the same acceleration --- their
different masses do not produce different trajectories.

The cancellation theorem of Sec.~\ref{sec:proof} therefore applies
to both degrees of freedom changed by a transition:

\textbf{Mass defect (internal).}
The direct pull $GM/R^2$ of the external body vanishes from the relative
acceleration between the atom and its environment (trap, lattice,
or mirror), regardless of the internal state. The mass defect
$\Delta m$ couples to the same gravitational field as the rest
mass, and that field undergoes the same cancellation.

\textbf{Recoil trajectory (external).}
After absorbing a photon, the atom follows a parabolic arc
$\vb{x}(t) = \vb{x}_0 + \vb{v}_r\, t + \tfrac{1}{2}\vb{g}\, t^2$
in the gravitational field. The acceleration~$\vb{g}$ governing
this arc is $g_0\,\hat{z} + \vb{a}_{\text{tidal}}$ --- it does not
include the direct pull $GM/R^2$, because the laboratory
(and hence the laser source that defined~$\vb{v}_r$) shares the
same free-fall acceleration. The recoil sets the initial
velocity; gravity steers the subsequent trajectory; and the
cancellation theorem ensures that both are referenced to~$g_0$
and tidal corrections only.

\subsection{Gravitational redshift as the surviving effect}

Although the direct pull cancels, the gravitational \emph{potential}
does affect the transition frequency. An atom at height~$h$ above
a reference point experiences a gravitational redshift:
\begin{equation}
\frac{\Delta\nu}{\nu_0}
  = \frac{\Phi(h) - \Phi(0)}{c^2}
  = \frac{g_0\, h}{c^2}
\label{eq:redshift}
\end{equation}
This is the standard result confirmed by Pound and Rebka (1960) and
measured to parts in $10^{18}$ by modern optical lattice clocks.
The gravitational acceleration entering Eq.~\eqref{eq:redshift}
is~$g_0$ --- the effective gravity including the centrifugal
correction, but \emph{not} the direct pull of the external body. The external body's
potential $\Phi_{\text{ext}} = -GM/R$ can be enormous
(e.g.\ $\Phi_{\text{ext}}/c^2 \approx -10^{-8}$ for the Sun), but it is spatially uniform
across any terrestrial laboratory and therefore contributes no
measurable frequency difference between two clocks at different
heights.

The only contribution of the external body to the redshift is tidal. At a
fixed location on Earth's surface, the tidal potential of the external body
oscillates semi-diurnally with amplitude
\begin{equation}
\frac{\delta\Phi_{\text{tidal}}}{c^2}
  \sim \frac{GM\, r_\oplus^2}{R^3\, c^2}
  \sim 2 \times 10^{-17}
\label{eq:tidal-redshift}
\end{equation}
as the Earth rotates through the tidal bulge. This is within
reach of the best optical lattice clocks
(${\sim}\,10^{-18}$ fractional uncertainty) and must be corrected
for in high-precision clock comparisons. A further contribution
of similar magnitude arises from the solid Earth tide, which
physically displaces the clock by ${\sim}\,30$~cm vertically,
shifting its position in Earth's own potential by
$g_0 \times 0.3\;\text{m}/c^2 \sim 3 \times 10^{-17}$.
Both effects are routinely modeled as systematic corrections in
precision metrology.

For a lab-scale height difference~$h$, the tidal redshift is
far smaller: $GM\, r_\oplus\, h / (R^3\, c^2)
\sim 3 \times 10^{-24}(h/1\;\text{m})$ --- well beyond current
sensitivity.

In all cases, the tidal redshift is the spectroscopic counterpart
of the tidal acceleration in Eq.~\eqref{eq:tidal}; the direct
pull $GM/R^2$ has no counterpart, for the same reason as in the
classical case.

\subsection{Implications for precision measurements}

The cancellation has concrete consequences for precision
experiments that use atoms as test masses. In each case, the
sensor realises the three-point geometry of
Appendix~\ref{app:three-point}, so the direct pull
$GM/R^2$ of the external body drops out of the observable; only $g_0$ and tidal
terms survive.

\textbf{Optical lattice clocks.}
No state-dependent force from the external body perturbs the transition frequency.
The lattice clock maps directly onto the three-point formulation:
housing $=$ lattice mirrors, spring $=$ trapping potential, test
mass $=$ atom. The oscillation amplitude
$\Delta x \sim 10$--$100$~nm suppresses tidal effects by
$\Delta x / r_\oplus \sim 10^{-14}$ relative to the
$r_\oplus$-scale signal, and the mass defect
$\Delta m = h\nu/c^2$ couples only to this already-negligible
residual.

\textbf{Atom interferometers.}
The measured acceleration $g$ in
$\Delta\phi = k_{\text{eff}}\, g\, T^2$ does not contain
$GM/R^2$. The atom is in free fall while the retroreflection
mirror is fixed to the ground --- a three-point geometry with no
spring. Each beam-splitter pulse imparts recoil
$\hbar\vb{k}_{\text{eff}}$ and changes the internal state,
creating two trajectories that share the same acceleration
$g_0 + a_{\text{tidal}}$. The mass defect enters through the
state-dependent recoil, Eq.~\eqref{eq:recoil}: the two arms
carry momenta $\hbar k_{\text{eff}} / m_g$ vs.\
$\hbar k_{\text{eff}} / m_e$, producing a differential phase of
order $h\nu / (m\, c^2) \sim 10^{-10}$ per fringe that couples
to $g_0$ alone and has been used to test $E = mc^2$ at the
atomic scale.

\textbf{Tests of the equivalence principle.}
An equivalence-principle violation at level~$\eta$ would couple
internal energy to gravity differently from rest mass, producing a
state-dependent frequency shift modulated at the orbital period.
For the Sun, Earth's orbital eccentricity ($e \approx 0.017$) varies the
external-body potential by
$\delta\Phi_{\text{ext}}/c^2 \sim e\, GM/(R\,c^2) \approx 1.7 \times 10^{-10}$,
so a violation would appear as a fractional frequency modulation
$\sim \eta \cdot 1.7 \times 10^{-10}$. Multi-year comparisons of
different atomic species (Rb/Cs microwave fountains, Sr and Yb
optical lattice clocks) have searched for this annual modulation
and currently constrain $\eta \lesssim 10^{-6}$--$10^{-7}$,
confirming that the mass defect gravitates universally at this
level. Next-generation optical clocks at $10^{-19}$ stability
could push toward $\eta \sim 10^{-9}$.

\subsection{Summary}

The cancellation of the direct gravitational pull is not limited
to classical springs. It extends to any local measurement of
gravitational coupling, including atomic transitions. Both
consequences of a transition --- the mass defect
$\Delta m = h\nu/c^2$ (internal) and the recoil
$\Delta\vb{p} = \hbar\vb{k}$ (external) --- couple to gravity
through the same field that undergoes cancellation.
Both internal states fall identically in the external body's field, recoiling
atoms follow arcs governed by $g_0$ alone, and no local
experiment --- spring, clock, or interferometer --- can
detect the direct pull $GM/R^2$. What survives, as in the
classical case, is the tidal field and the gravitational redshift
due to the local potential gradient~$g_0$. The equivalence
principle ensures that these are the only gravitational effects
accessible to a terrestrial laboratory.


\section{Frame hierarchy: what can be transformed away}
\label{app:frames}

The cancellation of the direct pull of external bodies (Theorem~\ref{thm:cancel})
removes $GM/R^2$ from the sensor reading. But it is natural to
ask: is this the end of the story, or can further gravitational
contributions be eliminated by an appropriate choice of reference
frame? The answer reveals a hierarchy of physical content.

\subsection{Three levels of elimination}

Consider a test mass at position~$\vb{r}$ relative to Earth's
center, subject to the local effective gravity~$g_0$ and the
tidal field~$\vb{a}_{\text{tidal}}$. Different frame choices
strip away different layers of the gravitational environment:

\textbf{Level 1: Geocentric frame.}
Transforming from the inertial frame to one co-moving with
Earth's center eliminates the direct pull $GM/R^2$ of any external body.
This is the cancellation of Sec.~\ref{sec:proof}. What
remains: $g_0$ (Earth's gravity plus centrifugal) and the tidal
field from all external bodies.

\textbf{Level 2: Freely falling frame at the test mass.}
Transforming to a frame in free fall at the test mass location
further eliminates~$g_0$ itself. In this frame, the
``gravitational force'' $m g_0$ disappears --- it was a
fictitious force arising from the non-inertial character of
the laboratory. What remains: only the tidal field, i.e.\ the
gradient of gravity across the finite extent of the apparatus.

\textbf{Level 3: No further elimination.}
The tidal field cannot be removed by any local frame choice.
It is encoded in the Riemann curvature tensor $R^\mu{}_{\nu\rho\sigma}$,
which is a tensor: if it is nonzero in one frame, it is nonzero
in all frames. Tidal effects are the irreducible gravitational
observable.

This hierarchy is the equivalence principle expressed operationally.
Each level eliminates a uniform contribution by recognizing it as
indistinguishable from an acceleration. Only the non-uniform
part --- curvature --- is frame-independent and therefore physical.

\subsection{Spring-based sensors}

A spring sensor holds the test mass at rest in the laboratory.
The test mass is \emph{not} freely falling: the spring force
provides a proper acceleration~$g_0$ that keeps it off its natural
geodesic. The sensor reading is precisely this proper acceleration.

In the freely falling frame (Level~2), the physics is transparent:
the laboratory accelerates upward at~$g_0$ and drags the spring
mount with it. The test mass, if released, would remain at rest
(on a geodesic). The spring force is what prevents this. It
measures the non-gravitational acceleration of the housing, not a
``gravitational pull'' on the mass. The only gravitational
information accessible to the spring is the tidal correction ---
the difference in free-fall acceleration between the housing and
the test mass location.

\subsection{Optical lattice clocks}

An atom trapped in a lattice potential is the quantum analog of
the spring sensor. The atom is held off its geodesic by the
trapping potential, which provides a proper acceleration~$g_0$.
In the freely falling frame at the atom, $g_0$ vanishes and the
lattice potential simply confines the atom in a non-gravitational
well.

The transition frequency is affected by gravity through the
redshift $\Delta\nu/\nu = g_0\, h/c^2$, where $h$ is the
height within the lattice. In the freely falling frame, this
redshift is reinterpreted as a Doppler shift: the lattice
(co-accelerating with the lab) moves relative to the local
inertial frame, producing the same frequency difference. The
physics is identical; only the description changes. The tidal
contribution to the redshift
($\sim GM\, r_\oplus^2 / R^3 c^2 \sim 10^{-17}$) survives
in all frames.

\subsection{Atom interferometers}

Atom interferometers occupy a unique position in this hierarchy:
the atom \emph{is} in free fall between laser pulses. It follows
a geodesic. In its rest frame (Level~2), the atom feels no
gravitational force at all --- neither $GM/R^2$ (eliminated at
Level~1) nor $g_0$ (eliminated at Level~2).

The interferometer phase $\Delta\phi = k_{\text{eff}}\, g\, T^2$
does not arise from a force on the atom. It arises because the
\emph{mirror} is not in free fall: the mirror accelerates at
$g_0$ upward (held by the Earth's crust), and each successive
laser pulse is emitted from a different position in the atom's
freely falling frame. The accumulated phase shift encodes the
relative acceleration between the freely falling atom and the
non-inertial mirror. In this picture, $g$ is the proper
acceleration of the mirror, not a force on the
atom~\cite{asenbaum2024}.

The tidal field enters as a correction: the atom's geodesic and
the mirror's worldline are separated by a distance that grows
during the interrogation time, and the tidal gradient across this
baseline produces a small deviation from the ideal
$k_{\text{eff}}\, g\, T^2$ scaling. For current atom
interferometers with baselines of order meters, this tidal
correction is of order $a_{\text{tidal}} \cdot T^2 \sim 10^{-7}
\times T^2$~m --- detectable in principle with long interrogation
times but far below the dominant $g_0\, T^2$ signal.

\subsection{Summary: what is real}

\begin{center}
\begin{tabular}{lcc}
\hline\hline
Contribution & Eliminated by & Physical? \\
\hline
Direct pull $GM/R^2$
  & Geocentric frame & No \\
Local gravity $g_0$
  & Freely falling frame & No$^*$ \\
Tidal field $\vb{a}_{\text{tidal}}$
  & Nothing (curvature) & Yes \\
\hline\hline
\end{tabular}
\end{center}

\noindent
${}^*$The reading $g_0$ of a spring sensor or lattice clock is
real as a measurement of proper acceleration (the sensor's
deviation from geodesic motion), but it is not a measurement of
a gravitational field. A freely falling observer at the same
location would measure zero. Only the tidal field is
unambiguously gravitational: it cannot be mimicked or removed by
any choice of motion.

This hierarchy underscores the central message: the direct
pull of a distant body is not merely ``canceled'' by a fortunate subtraction. It is
\emph{unphysical} in the same sense that $g_0$ is unphysical ---
both are frame-dependent artifacts that can be transformed away.
The only difference is that $g_0$ requires a freely falling frame
to eliminate, while the direct pull is already absent in the
geocentric frame. What remains in all frames and for all
sensors --- spring, clock, or interferometer --- is the tidal
field, the sole irreducible gravitational observable.

\begin{thebibliography}{9}

\bibitem{asenbaum2024}
P.~Asenbaum, C.~Overstreet, and M.~A.~Kasevich,
``Matter waves and clocks do not observe uniform gravitational fields,''
Phys.\ Scr.\ \textbf{99}, 046103 (2024).

\end{thebibliography}

\end{document}
