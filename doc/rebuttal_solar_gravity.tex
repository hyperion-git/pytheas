\documentclass[aps,prx,onecolumn,superscriptaddress,nofootinbib]{revtex4-2}

\usepackage{amsmath,amssymb}
\usepackage{booktabs}
\usepackage{graphicx}
\usepackage{hyperref}
\usepackage{xcolor}

\hypersetup{colorlinks=true, linkcolor=blue!60!black, urlcolor=blue!60!black}

\newcommand{\vb}[1]{\mathbf{#1}}
\newcommand{\vu}[1]{\vb{e}_{\vb{#1}}}

\begin{document}

\title{Falling Together: Why Springs Can't See the Sun}
\author{Alexander Friedrich}
\affiliation{DLR Institute of Quantum Technologies, Wilhelm-Runge-Str.\ 10, 89081 Ulm, Germany}
\date{\today}

\begin{abstract}
A common claim holds that a sensor tilted relative to the local
vertical can detect the direct gravitational pull of the Sun
(${\sim}\,6 \times 10^{-3}$~m/s$^2$) or the Moon
(${\sim}\,3.3 \times 10^{-5}$~m/s$^2$) as a time-varying signal.
This is incorrect. We show, using only Newtonian point-mass mechanics,
that the direct pull cancels exactly against the acceleration of the
laboratory, leaving only the tidal residual --- regardless of sensor
orientation. The proof is given in an inertial frame where no terms are
hidden, and the result is then transformed to the rotating laboratory
frame.
\end{abstract}

\maketitle


\section{Introduction}
\label{sec:frames}

Can a spring-based accelerometer measure the Sun's direct
gravitational pull on its test mass? This note shows that the
answer is \textbf{no} --- not because the pull is too small, but
because the equivalence principle forbids it. The Sun pulls
the sensor housing and the test mass by exactly the same amount;
the spring connecting them cannot register a force that acts
identically on both ends.

The difficulty in seeing this lies in the choice of reference
frame. A laboratory on Earth's surface is non-inertial for two
reasons:

\begin{enumerate}
\item \textbf{Earth rotates}, producing centrifugal and Coriolis
  pseudo-accelerations (${\sim}\,0.03$~m/s$^2$ at the equator).

\item \textbf{Earth freely falls toward the Sun and Moon.} The
  sensor housing is rigidly attached to the Earth's crust; it
  shares the Earth's free-fall acceleration toward the Sun
  ($GM_S/R^2 \approx 5.93 \times 10^{-3}$~m/s$^2$) and Moon
  ($GM_M/D^2 \approx 3.32 \times 10^{-5}$~m/s$^2$) exactly.
  The test mass, connected only by a spring, is the one body
  free to deviate --- but gravity accelerates it by the same
  amount, so the spring registers no difference.
\end{enumerate}

\noindent
The second point is the origin of the cancellation. When one
writes equations of motion in the lab frame, the free-fall
acceleration toward the Sun has already been subtracted from every
term. A spring-based sensor physically implements this
subtraction: both ends share the same gravitational acceleration,
so the spring reads zero from that field. Only the tiny
\emph{gradient} of the field across the sensor --- the tidal
term --- survives.

To make this cancellation explicit rather than hidden, we begin in
a true inertial frame and carry every term (Sec.~\ref{sec:proof}).
We then transform to the rotating laboratory
(Sec.~\ref{sec:lab}), project onto an arbitrary sensor axis
(Sec.~\ref{sec:orientation}), and compare the magnitudes of all
surviving terms (Sec.~\ref{sec:observable}).
Section~\ref{sec:ep} places the result in the context of the
equivalence principle.


\section{The Proof}
\label{sec:proof}

\subsection{Setup}

Four point masses in an inertial frame (Fig.~\ref{fig:setup}).
The test mass is connected to the sensor by a spring (or equivalent
restoring mechanism). The sensor housing is rigidly attached to the
Earth. The sensor reads the spring force~$\vb{F}_{\text{spring}}$ ---
the non-gravitational force required to keep the test mass co-moving
with the lab.

\begin{figure}[h]
\centering
\begin{minipage}[c]{0.30\columnwidth}
\centering
\begin{tabular}{lll}
\toprule
Body & Mass & Position \\
\midrule
Sun       & $M_S$ & $\vb{R}_S(t)$ \\
Moon      & $M_M$ & $\vb{R}_M(t)$ \\
Earth     & $M_E$ & $\vb{R}_E(t)$ \\
Test mass & $m$   & $\vb{x}(t)$ \\
\bottomrule
\end{tabular}
\end{minipage}%
\hfill
\begin{minipage}[c]{0.67\columnwidth}
\centering
\includegraphics[width=\linewidth]{figures/setup_schematic.pdf}
\end{minipage}
\caption{Four-body setup. \emph{Left}: masses and inertial-frame
positions. \emph{Right}: schematic geometry. Gray dashed arrows:
position vectors $\vb{R}_S$, $\vb{R}_E$, $\vb{R}_M$,
and~$\vb{x}$ from the inertial origin. Red solid arrows:
geocentric vectors $\vb{R} = \vb{R}_S - \vb{R}_E$,
$\vb{D} = \vb{R}_M - \vb{R}_E$, and
$\vb{r} = \vb{x} - \vb{R}_E$. The spring connects the test
mass~$m$ to the Earth-fixed sensor housing.}
\label{fig:setup}
\end{figure}

\subsection{Test mass equation of motion}

In the inertial frame, Newton's second law for the test mass:
\begin{equation}
m\,\ddot{\vb{x}}
  = -\frac{G M_E\, m}{\lvert\vb{x} - \vb{R}_E\rvert^3}
    (\vb{x} - \vb{R}_E)
  \;-\frac{G M_S\, m}{\lvert\vb{x} - \vb{R}_S\rvert^3}
    (\vb{x} - \vb{R}_S)
  \;-\frac{G M_M\, m}{\lvert\vb{x} - \vb{R}_M\rvert^3}
    (\vb{x} - \vb{R}_M)
  \;+\; \vb{F}_{\text{spring}}
\label{eq:test}
\end{equation}

\subsection{Earth's equation of motion}

Earth's center accelerates toward the Sun and Moon:
\begin{equation}
\ddot{\vb{R}}_E
  = -\frac{G M_S}{\lvert\vb{R}_E - \vb{R}_S\rvert^3}
    (\vb{R}_E - \vb{R}_S)
  \;-\; \frac{G M_M}{\lvert\vb{R}_E - \vb{R}_M\rvert^3}
    (\vb{R}_E - \vb{R}_M)
\label{eq:earth}
\end{equation}

Every object rigidly attached to the Earth --- including the sensor
housing --- shares this acceleration.

\subsection{Subtract to get the relative motion}

Define $\vb{r} \equiv \vb{x} - \vb{R}_E$ and the geocentric positions
$\vb{R} \equiv \vb{R}_S - \vb{R}_E$,
$\vb{D} \equiv \vb{R}_M - \vb{R}_E$. Then
$\ddot{\vb{r}} = \ddot{\vb{x}} - \ddot{\vb{R}}_E$ gives:
\begin{equation}
m\,\ddot{\vb{r}}
  = -\frac{G M_E\, m}{\lvert\vb{r}\rvert^3}\,\vb{r}
  \;+\; G M_S\, m \left[
    \frac{\vb{R} - \vb{r}}{\lvert\vb{R} - \vb{r}\rvert^3}
    - \frac{\vb{R}}{R^3}
  \right]
  \;+\; G M_M\, m \left[
    \frac{\vb{D} - \vb{r}}{\lvert\vb{D} - \vb{r}\rvert^3}
    - \frac{\vb{D}}{D^3}
  \right]
  \;+\; \vb{F}_{\text{spring}}
\label{eq:relative}
\end{equation}

\subsection{The cancellation}

Each bracketed term has the form
\begin{equation}
\vb{f}(\vb{r})
  = \frac{\vb{R} - \vb{r}}{\lvert\vb{R} - \vb{r}\rvert^3}
  - \frac{\vb{R}}{R^3}
\end{equation}

This is the Sun's gravitational field at the test mass \emph{minus} the
field at Earth's center. At $\vb{r} = \vb{0}$:
\begin{equation}
\vb{f}(\vb{0})
  = \frac{\vb{R}}{R^3} - \frac{\vb{R}}{R^3}
  = \vb{0}
  \qquad\text{(exactly)}
\label{eq:cancel}
\end{equation}

The direct pull --- $GM_S\,\vb{R}/R^3$, the uniform field that
accelerates the entire Earth at $5.93 \times 10^{-3}$~m/s$^2$ ---
appears with equal magnitude and opposite sign in
Eqs.~\eqref{eq:test} and~\eqref{eq:earth}, and cancels identically.
No approximation is involved. The same holds for the Moon with $\vb{D}$
replacing~$\vb{R}$.

What survives is $\vb{f}(\vb{r}) - \vb{f}(\vb{0})$: the
\emph{variation} of the gravitational field across the
baseline~$\vb{r}$ --- the \textbf{tidal acceleration}. To leading order
in $r/R$:
\begin{equation}
\vb{a}_{\text{tidal}}
  = \frac{GM}{R^3}\Big[
    3\,(\vu{R}\cdot\vb{r})\,\vu{R} - \vb{r}
  \Big]
  + O\!\left(\frac{GM\,r^2}{R^4}\right)
\label{eq:tidal}
\end{equation}

This is suppressed by a factor $r/R$ relative to the direct pull.

\subsection{Derivation of Eq.~\eqref{eq:tidal}}

Expand $\lvert\vb{R} - \vb{r}\rvert^{-3}$ for $r \ll R$. Write
\begin{equation}
\lvert\vb{R} - \vb{r}\rvert^2
  = R^2\!\left(1
    - 2\,\frac{\vu{R}\cdot\vb{r}}{R}
    + \frac{r^2}{R^2}\right)
  \equiv R^2(1 - \epsilon)
\end{equation}
with $\epsilon = 2\,\vu{R}\cdot\vb{r}/R - r^2/R^2 = O(r/R)$. Then
\begin{equation}
\frac{1}{\lvert\vb{R} - \vb{r}\rvert^3}
  = \frac{1}{R^3}(1-\epsilon)^{-3/2}
  = \frac{1}{R^3}\!\left(1
    + \frac{3\,\vu{R}\cdot\vb{r}}{R}
    + O\!\left(\frac{r^2}{R^2}\right)\right)
\end{equation}

Multiply by $(\vb{R} - \vb{r})$ and keep terms through first order:
\begin{equation}
\frac{\vb{R} - \vb{r}}{\lvert\vb{R} - \vb{r}\rvert^3}
  = \frac{1}{R^3}\Big[
    \vb{R} + 3(\vu{R}\cdot\vb{r})\,\vu{R} - \vb{r}
  \Big]
  + O\!\left(\frac{r^2}{R^4}\right)
\end{equation}

Subtract $\vb{R}/R^3$:
\begin{equation}
\vb{f}(\vb{r})
  = \frac{\vb{R} - \vb{r}}{\lvert\vb{R} - \vb{r}\rvert^3}
  - \frac{\vb{R}}{R^3}
  = \frac{1}{R^3}\Big[
    3(\vu{R}\cdot\vb{r})\,\vu{R} - \vb{r}
  \Big]
  + O\!\left(\frac{r^2}{R^4}\right)
\end{equation}

The $\vb{R}/R^3$ terms cancel identically --- confirming that the
direct pull drops out --- and the leading surviving term is the tidal
quadrupole, Eq.~\eqref{eq:tidal}.\hfill$\square$


\section{Transformation to the Laboratory Frame}
\label{sec:lab}

Equation~\eqref{eq:relative} is written in the non-rotating geocentric
frame. The laboratory rotates with the Earth at angular
velocity~$\boldsymbol{\omega}$. This is a second source of
non-inertiality (see Sec.~\ref{sec:frames}).

\subsection{Rotating-frame equation of motion}

Let $\vb{r}'$ be the test mass position in the lab frame, related to
its geocentric position by
$\vb{r} = \mathcal{R}(t)\,\vb{r}'$
where $\mathcal{R}(t)$ is the time-dependent rotation matrix. For
uniform rotation about axis
$\hat{\vb{n}} = \boldsymbol{\omega}/\omega$ at angular
rate $\omega = \lvert\boldsymbol{\omega}\rvert$, it takes the
Rodrigues form
\begin{equation}
\mathcal{R}(t)
  = \vb{I}\cos\omega t
  + (1 - \cos\omega t)\,\hat{\vb{n}}\,\hat{\vb{n}}^T
  + \sin\omega t\;[\hat{\vb{n}}]_\times
\label{eq:rodrigues}
\end{equation}
where $[\hat{\vb{n}}]_\times$ is the skew-symmetric matrix whose
action on any vector~$\vb{v}$ is
$[\hat{\vb{n}}]_\times\vb{v} = \hat{\vb{n}} \times \vb{v}$
--- i.e.\ the matrix representation of the cross product. Its time
derivative satisfies
\begin{equation}
\dot{\mathcal{R}}(t)
  = [\boldsymbol{\omega}]_\times\;\mathcal{R}(t)
\label{eq:Rdot}
\end{equation}
so that
$\dot{\mathcal{R}}\,\vb{v}
  = \boldsymbol{\omega} \times (\mathcal{R}\,\vb{v})$
for every vector~$\vb{v}$. Each time derivative
of~$\mathcal{R}$ thus generates a cross product
with~$\boldsymbol{\omega}$.

Differentiating $\vb{r} = \mathcal{R}\,\vb{r}'$ twice using
Eq.~\eqref{eq:Rdot}:
\begin{equation}
\ddot{\vb{r}}
  = \ddot{\vb{r}}'
  + 2\,\boldsymbol{\omega} \times \dot{\vb{r}}'
  + \boldsymbol{\omega} \times (\boldsymbol{\omega} \times \vb{r}')
\label{eq:kinematic}
\end{equation}
(the Euler term $\dot{\boldsymbol{\omega}} \times \vb{r}'$ vanishes for
uniform rotation). Note that $\boldsymbol{\omega}$ carries no prime:
$\mathcal{R}(t)$ is a rotation \emph{about}~$\boldsymbol{\omega}$,
so $\mathcal{R}^{-1}\boldsymbol{\omega} = \boldsymbol{\omega}$ ---
the rotation vector is invariant under the rotation it generates.
Substituting into Eq.~\eqref{eq:relative} and
expressing all vectors in the lab basis:
\begin{equation}
\ddot{\vb{r}}'
  = \underbrace{-\frac{G M_E}{\lvert\vb{r}'\rvert^3}\,\vb{r}'
    }_{\text{Earth's gravity}}
  \;+\; \underbrace{\vb{a}_{\text{tidal}}'
    }_{\text{tidal}}
  \;\underbrace{-\; \boldsymbol{\omega}
    \times (\boldsymbol{\omega} \times \vb{r}')
    }_{\text{centrifugal}}
  \;\underbrace{-\; 2\,\boldsymbol{\omega}
    \times \dot{\vb{r}}'
    }_{\text{Coriolis}}
  \;+\; \frac{\vb{F}_{\text{spring}}'}{m}
\label{eq:lab}
\end{equation}

The direct solar and lunar pulls do \textbf{not} reappear --- they
canceled in Eq.~\eqref{eq:relative} before the frame change, and
rotating the coordinate basis cannot restore a term that is already
zero.

\subsection{What the sensor reads}

The sensor housing is held in place by the laboratory floor, which
provides whatever normal force~$\vb{N}$ is required to prevent
acceleration. This constraint force acts on the housing, not on the
test mass, and therefore does not appear in Eq.~\eqref{eq:lab}. Its
effect enters through the spring: the housing end of the spring
follows the Earth's motion, so the spring
force~$\vb{F}_{\text{spring}}'$ adjusts to keep the test mass
co-moving.

The spring force is thus entirely \emph{caused by} the constraint.
The floor holds the housing fixed; the spring transmits that
constraint to the test mass. This is the equivalence principle in
action: if the floor were removed and the entire apparatus fell
freely, housing and test mass would share the same gravitational
acceleration. The spring would relax and
$\vb{F}_{\text{spring}}' \to \vb{0}$ --- precisely because no
constraint force distinguishes the housing from the test mass.
A spring-based sensor can only measure accelerations that
\emph{differ} between its two ends, i.e.\ tidal effects and
non-gravitational forces.

The test mass is thus at rest in the lab:
$\dot{\vb{r}}' = \ddot{\vb{r}}' = \vb{0}$. The Coriolis term
vanishes. Setting the left-hand side of Eq.~\eqref{eq:lab} to zero
determines the spring force --- the quantity the sensor reads:
\begin{equation}
\frac{\vb{F}_{\text{spring}}'}{m}
  = \frac{G M_E}{\lvert\vb{r}'\rvert^3}\,\vb{r}'
  + \boldsymbol{\omega} \times (\boldsymbol{\omega} \times \vb{r}')
  - \vb{a}_{\text{tidal}}'
\label{eq:spring}
\end{equation}

The sensor projects this onto its measurement axis~$\vu{n}$:
\begin{equation}
g_{\text{measured}}
  = \frac{\vb{F}_{\text{spring}}'}{m} \cdot \vu{n}
\end{equation}

The three contributions are:
\begin{itemize}
\item $G M_E\,\vb{r}'/\lvert\vb{r}'\rvert^3$: Earth's gravity (the
  dominant, static term).
\item $\boldsymbol{\omega}\times(\boldsymbol{\omega}\times\vb{r}')
  = -\omega^2\vb{r}'_\perp$: the centrifugal reduction (objects weigh
  less at the equator by ${\sim}\,0.3\%$).
\item $-\vb{a}_{\text{tidal}}'$: the tidal perturbation from Sun and
  Moon, of order $10^{-7}$~m/s$^2$.
\end{itemize}

\noindent
The direct pull $GM_S/R^2$ and $GM_M/D^2$ appear nowhere. This holds
for any~$\vu{n}$.

\subsection{Size of each term}

Table~\ref{tab:terms} lists every acceleration that enters
Eq.~\eqref{eq:spring}, together with the two direct pulls that
cancel before reaching it.

\begin{table}[h]
\caption{Magnitude of each acceleration term at Earth's surface,
ordered by size. The grayed-out rows --- the direct gravitational
pulls --- cancel exactly (Sec.~\ref{sec:proof}) and do not appear in
the sensor reading [Eq.~\eqref{eq:spring}]. The last column gives
the ratio of each term to the solar tidal residual
($5.0 \times 10^{-7}$~m/s$^2$); for the canceled rows it is the
ratio of direct pull to corresponding tidal residual ($11{,}700\times$
for the Sun, $30\times$ for the Moon).}
\label{tab:terms}
\begin{ruledtabular}
\begin{tabular}{llrrr}
\textsc{Term} & \textsc{Expression}
  & \textsc{Magnitude} (m/s$^2$)
  & \textsc{Daily variation} (m/s$^2$)
  & \textsc{Ratio} \\
\hline
Earth's gravity
  & $GM_E/r_\oplus^2$
  & $9.82$
  & static
  & $2 \times 10^7$ \\
Centrifugal ($\lambda\!=\!48.4°$)
  & $\omega^2 r_\oplus \cos^2\!\lambda$
  & $3.4 \times 10^{-2}\cos^2\!\lambda$
  & static
  & $7 \times 10^{4}\cos^2\!\lambda$ \\
\textcolor{gray}{Solar direct pull (canceled)}
  & \textcolor{gray}{$GM_S / R^2$}
  & \textcolor{gray}{$5.93 \times 10^{-3}$}
  & \textcolor{gray}{---}
  & \textcolor{gray}{$11{,}700\times$} \\
\textcolor{gray}{Lunar direct pull (canceled)}
  & \textcolor{gray}{$GM_M / D^2$}
  & \textcolor{gray}{$3.32 \times 10^{-5}$}
  & \textcolor{gray}{---}
  & \textcolor{gray}{$30\times$} \\
Lunar tide
  & $GM_M\, r_\oplus / D^3$
  & $1.1 \times 10^{-6}$
  & ${\lesssim}\; 2 \times 10^{-6}$
  & --- \\
Solar tide
  & $GM_S\, r_\oplus / R^3$
  & $5.0 \times 10^{-7}$
  & ${\lesssim}\; 8 \times 10^{-7}$
  & --- \\
\end{tabular}
\end{ruledtabular}
\end{table}

The hierarchy is striking. Earth's gravity dominates at
${\sim}\,10$~m/s$^2$. The centrifugal correction is four orders of
magnitude smaller (${\sim}\,10^{-2}$~m/s$^2$), yet still measurable ---
it is the reason objects weigh less at the equator than at the poles.
The tidal accelerations are smaller still, at $10^{-6}$--$10^{-7}$~m/s$^2$,
but are routinely detected by superconducting gravimeters and satellite
missions such as GRACE.

The canceled direct pulls occupy a revealing intermediate scale.
The Sun's direct pull ($5.93 \times 10^{-3}$~m/s$^2$) is
$180\times$ stronger than the Moon's ($3.32 \times 10^{-5}$~m/s$^2$),
yet its tidal effect is $2.2\times$ \emph{weaker}
($1/R^3$ vs.\ $1/R^2$ scaling). This inversion --- the Sun pulls
harder but tidally disturbs less --- is a direct signature of the
cancellation. If direct pulls were observable, the Sun would dominate
the signal by two orders of magnitude. Instead, the Moon dominates
the tidal signal, exactly as observed.

The sensor reads a superposition of the four non-grayed terms in
Table~\ref{tab:terms}. The direct pulls have been subtracted out by
the physics of the measurement: the lab accelerates with the Earth,
and the spring cannot see it.


\section{Why Sensor Orientation Is Irrelevant}
\label{sec:orientation}

The cancellation in Eq.~\eqref{eq:cancel} is \textbf{vectorial}: the
direct pull vanishes as a three-component vector, not as a particular
scalar projection. Projecting onto any measurement axis~$\vu{n}$:
\begin{equation}
\vu{n} \cdot \vb{f}(\vb{0}) = \vu{n} \cdot \vb{0} = 0
  \qquad\text{for all } \vu{n}
\end{equation}

Tilting the sensor changes~$\vu{n}$ but cannot make zero non-zero.
The spring force along any axis contains only the tidal terms from
Eq.~\eqref{eq:tidal}.

Physically: the sensor housing and the test mass both accelerate at
$GM_S/R^2$ toward the Sun. The spring connecting them measures only the
\emph{difference} between the accelerations of its two endpoints. A
uniform field produces no difference, regardless of the spring's
orientation.

\subsection{Sweeping the sensor orientation}

Parameterise the measurement axis by zenith angle~$\theta$ and
azimuth~$\varphi$:
\begin{equation}
\vu{n}(\theta,\varphi)
  = \sin\theta\cos\varphi\;\vu{E}
  + \sin\theta\sin\varphi\;\vu{N}
  + \cos\theta\;\vu{U}
\end{equation}
where $\vu{E}$, $\vu{N}$, $\vu{U}$ are the local East--North--Up
unit vectors defined relative to the \emph{geodetic} vertical.
Projecting Eq.~\eqref{eq:spring} onto $\vu{n}(\theta,\varphi)$
and writing each term explicitly:
\begin{align}
g(\theta,\varphi)
  &= \underbrace{\frac{GM_E}{r_\oplus^2}\cos\theta
       \;+\; \bigl[\boldsymbol{\omega}\times
       (\boldsymbol{\omega}\times\vb{r}')\bigr]
       \cdot\vu{n}}_{\text{gravity + centrifugal}}
  \;+\; \underbrace{(-2\boldsymbol{\omega}
       \times\dot{\vb{r}}')\cdot\vu{n}}_{=\,0\;\text{(static)}}
  \;-\; \vb{a}_{\text{tidal}}'\cdot\vu{n}
  \notag \\[6pt]
  &= g_0\cos\theta
  \;-\; a_E'\sin\theta\cos\varphi
  \;-\; a_N'\sin\theta\sin\varphi
  \;-\; a_U'\cos\theta
\label{eq:sweep}
\end{align}
In the first line, the gravitational and centrifugal projections
combine into a single $\cos\theta$ term because both point along
$\vu{U}$, and the Coriolis term vanishes for a stationary test
mass ($\dot{\vb{r}}' = \vb{0}$). The result defines the effective
gravity $g_0 \equiv GM_E/r_\oplus^2
- \omega^2 r_\oplus\cos^2\!\lambda$ (where $\lambda$ is the geodetic
latitude), absorbing the centrifugal correction into the amplitude.

We now trace each contribution through this projection.

\textbf{Earth's gravity and centrifugal ($g_0\cos\theta$).}
The effective gravity
$\vb{g}_{\text{eff}} = g_0\,\vu{U}$ points purely along the
geodetic vertical by definition.
Its projection onto the sensor axis is $g_0\cos\theta$: it reads
the full $g_0 \approx 9.81$~m/s$^2$ when the sensor is vertical
($\theta = 0$), vanishes when horizontal ($\theta = 90°$), and
reverses when inverted ($\theta = 180°$). The centrifugal correction
modifies~$g_0$ by at most $0.3\%$ (latitude-dependent) but
introduces no separate angular signature --- it is entirely absorbed
into the amplitude of the $\cos\theta$ term. There is no
``rotational residual'' in the sweep.

\textbf{Solar and lunar tides ($a_E'$, $a_N'$, $a_U'$).}
The tidal accelerations from Eq.~\eqref{eq:tidal} decompose into
East, North, and Up components. The horizontal components $a_E'$
and $a_N'$ enter through $\sin\theta\cos\varphi$ and
$\sin\theta\sin\varphi$: they vanish for a vertical sensor and are
maximized for a horizontal one pointed in the appropriate direction.
The vertical component $a_U'$ enters as $\cos\theta$, superimposed
on the much larger $g_0\cos\theta$. Representative combined
amplitudes (from Sec.~\ref{sec:lab}): solar tidal
${\sim}\,5.0 \times 10^{-7}$~m/s$^2$, lunar tidal
${\sim}\,1.1 \times 10^{-6}$~m/s$^2$. These are the only
time-varying signals in the sweep.

\textbf{Direct solar pull (canceled).}
Were the direct pull $GM_S/R^2$ not canceled by Earth's free fall,
it would project as
$a_\odot\sin\theta\cos(\varphi - \varphi_\odot)$
with amplitude $a_\odot \approx 5.93 \times 10^{-3}$~m/s$^2$ ---
four orders of magnitude above the tidal terms. As shown in
Fig.~\ref{fig:sweep}, no such signal exists at any orientation.
An angular scan therefore confirms the cancellation directly: one
measures $g_0\cos\theta$ plus tidal corrections of order
$10^{-7}$~m/s$^2$, with no $10^{-3}$~m/s$^2$ sinusoidal component.

\begin{figure}[t]
\centering
\includegraphics[width=\columnwidth]{figures/orientation_sweep.pdf}
\caption{Residual acceleration $g - g_\mathrm{grav}\cos\theta$ as a function of
sensor orientation, where $g_\mathrm{grav}$ is the pure gravitational
acceleration (without centrifugal). Left column: sweep over zenith
angle~$\theta$ at fixed azimuth $\varphi = 0°$ (East). Right column:
sweep over azimuth~$\varphi$ at fixed $\theta = 45°$.
\textbf{Top row}~(a,\,b): $10^{-3}$~m/s$^2$ scale.
Solid blue: centrifugal correction
($\omega^2 r_\oplus \cos^2\!\lambda \approx 1.7 \times 10^{-2}$~m/s$^2$
at Ulm, $\lambda = 48.4°$);
red dashed: hypothetical Sun direct pull
($5.93 \times 10^{-3}$~m/s$^2$);
orange dashed: hypothetical Moon direct pull
($3.3 \times 10^{-5}$~m/s$^2$);
blue solid: combined tidal residual (indistinguishable from zero at this
scale). The centrifugal term is the largest non-gravitational contribution
and exceeds even the (canceled) Sun direct pull.
\textbf{Bottom row}~(c,\,d): $10^{-7}$~m/s$^2$ scale, zoomed to the
actual tidal signals. Teal: solar tidal; magenta: lunar tidal;
blue: combined. The centrifugal ($10^5\times$), canceled direct Sun
($10^4\times$), and Moon ($30\times$) pulls are all off this scale.
No orientation reveals the claimed $10^{-3}$~m/s$^2$ signal.}
\label{fig:sweep}
\end{figure}


\section{What IS Observable}
\label{sec:observable}

Given that the direct pulls cancel, what \emph{does} a tilted
sensor actually measure? From Eq.~\eqref{eq:sweep}, the sensor
reading decomposes into three physically distinct contributions.
Each responds differently to changes in orientation and time.

\subsection{Static baseline: gravity and centrifugal}

The dominant term $g_0\cos\theta$ is the projection of the
effective gravity onto the measurement axis. It is entirely static
--- independent of time, the positions of Sun and Moon, or
anything else in the sky. Tilting the sensor from vertical
($\theta = 0$) to horizontal ($\theta = 90°$) sweeps this term
from $g_0 \approx 9.81$~m/s$^2$ through zero.

Embedded within $g_0$ is the centrifugal correction, which at Ulm
amounts to $\omega^2 r_\oplus\cos^2\!\lambda \approx
1.5 \times 10^{-2}$~m/s$^2$. It projects as
$1.5 \times 10^{-2}\cos\theta$~m/s$^2$ and is visible in
Fig.~\ref{fig:sweep} (panels~a,\,b) as the solid blue curve.
This is the largest non-gravitational contribution the sensor
can detect --- it exceeds even the (canceled) solar direct pull
by a factor of~$2.5$. Unlike the tidal terms, however, it carries
no time dependence: it is a fixed geometric projection that shifts
the baseline reading as the sensor is tilted.

\subsection{Time-varying signals: tidal components}

The only orientation-dependent terms that vary in time are the
tidal projections $a_E'\sin\theta\cos\varphi$,
$a_N'\sin\theta\sin\varphi$, and $a_U'\cos\theta$. These
encode the three components of the tidal acceleration vector
in the geodetic frame:

\begin{itemize}
\item A vertical sensor ($\theta = 0$) measures $a_U'$, the
  vertical tidal component (${\sim}\,5.0 \times 10^{-7}$~m/s$^2$,
  semi-diurnal with ${\sim}\,12$~h period).
\item A horizontal sensor ($\theta = 90°$) pointing East
  ($\varphi = 0$) measures $a_E'$; pointing North
  ($\varphi = 90°$) it measures $a_N'$. These horizontal
  components are diurnal (${\sim}\,24$~h period).
\item Three non-coplanar sensors reconstruct the full tidal
  acceleration vector and, in principle, the tidal tensor.
\end{itemize}

\noindent
The lunar tidal signal (${\sim}\,1.1 \times 10^{-6}$~m/s$^2$) is
roughly twice the solar (${\sim}\,5.0 \times 10^{-7}$~m/s$^2$),
consistent with the $1/R^3$ scaling that favors the nearby Moon
over the distant Sun. Superconducting gravimeters routinely resolve
these signals; they are the standard observable in terrestrial
tidal gravimetry.

\subsection{What is absent}

Conspicuously absent from Eq.~\eqref{eq:sweep} are two terms
that would dominate the reading if direct gravitational pulls
were not canceled:

\begin{itemize}
\item A \textbf{solar} term of order
  $5.93 \times 10^{-3}$~m/s$^2$, projecting as
  $a_\odot\sin\theta\cos(\varphi - \varphi_\odot)$ with a
  diurnal period (${\sim}\,24$~h). This would exceed the
  actual tidal signals by four orders of magnitude.
\item A \textbf{lunar} term of order
  $3.32 \times 10^{-5}$~m/s$^2$, projecting as
  $a_M\sin\theta\cos(\varphi - \varphi_M)$
  with a period of ${\sim}\,24.8$~h. Smaller than the solar
  term by a factor of~$180$, but still $30\times$ larger than
  the lunar tidal signal it would accompany.
\end{itemize}

\noindent
Figure~\ref{fig:sweep} confirms their absence directly: the top
panels show both hypothetical direct pulls as dashed curves that
dwarf the actual tidal residual (blue, indistinguishable from
zero at this scale). The bottom panels zoom into the
$10^{-7}$~m/s$^2$ scale where the real tidal signals live ---
orders of magnitude below either claimed effect.

All orientation-dependent signals the sensor can detect ---
centrifugal and tidal --- are real and well understood.
None is a direct gravitational pull.


\section{The Equivalence Principle}
\label{sec:ep}

The cancellation derived in Sec.~\ref{sec:proof} is not a
calculational coincidence. It is the equivalence principle.

Einstein's key insight was that a uniform gravitational field is
locally indistinguishable from an accelerating reference frame.
No experiment confined to a sufficiently small laboratory can
determine whether the laboratory is at rest in a gravitational
field or accelerating through empty space. This means that a
freely falling observer cannot detect the uniform component of any
gravitational field --- only its \emph{gradient} (the tidal part)
is locally measurable.

The Earth, together with the sensor, the laboratory, and the
observer, is in free fall in the Sun's gravitational field. The
entire system accelerates at $GM_S/R^2$ toward the Sun. A
spring-based sensor measures the force difference between its two
endpoints. Because both endpoints share the same free-fall
acceleration, the spring is blind to the Sun's uniform pull. Only
the \emph{non-uniformity} of the field across the sensor ---
the tidal term, of order $GM_S\,r_\oplus/R^3$ --- can produce a
differential signal.

In the language of general relativity, the Earth follows a geodesic
in the Sun's spacetime. Along a geodesic, no proper acceleration
is felt --- the ``force of gravity'' is replaced by the geometry
of spacetime. What remains observable is geodesic deviation: nearby
geodesics converge or diverge due to spacetime curvature. This is
the tidal acceleration of Eq.~\eqref{eq:tidal}, encoded in the
Riemann tensor. The direct pull $GM_S/R^2$ has no counterpart in
this description; it is an artifact of the Newtonian decomposition
into ``gravitational force'' and ``inertial motion,'' a
decomposition that the equivalence principle declares unphysical.

This is why no adjustment of sensor orientation, no change of
measurement axis, and no increase in sensitivity can reveal a
$5.93 \times 10^{-3}$~m/s$^2$ signal from the Sun. The signal does
not exist at the sensor because the equivalence principle forbids
it. What the sensor can measure --- and what is measured routinely
--- are tidal effects at $10^{-7}$--$10^{-6}$~m/s$^2$, four orders
of magnitude below the direct pull. The proof in
Sec.~\ref{sec:proof} is the Newtonian statement of this principle;
the equivalence principle elevates it from a property of gravity to
a law of nature.

\end{document}
