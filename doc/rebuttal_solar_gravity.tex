\documentclass[11pt,a4paper]{article}

\usepackage[utf8]{inputenc}
\usepackage[T1]{fontenc}
\usepackage{lmodern}
\usepackage{amsmath,amssymb}
\usepackage{booktabs}
\usepackage{geometry}
\usepackage{hyperref}
\usepackage{xcolor}

\geometry{margin=2.5cm}
\hypersetup{colorlinks=true, linkcolor=blue!60!black, urlcolor=blue!60!black}

\newcommand{\vb}[1]{\mathbf{#1}}
\newcommand{\vu}[1]{\vb{e}_{\vb{#1}}}

\title{Why a Tilted Sensor Cannot Measure\\Direct Gravitational Pull}
\author{}
\date{}

\begin{document}
\maketitle

\section*{The Claim}

\begin{quote}
A sensor tilted relative to the local vertical can detect the direct
gravitational pull of the Sun (${\sim}\,6 \times 10^{-3}$~m/s$^2$) or the
Moon (${\sim}\,3.3 \times 10^{-5}$~m/s$^2$) as a time-varying signal,
because the projection of the gravitational acceleration onto the sensor
axis changes as these bodies move across the sky.
\end{quote}

\noindent
This is incorrect. The direct pull cancels exactly against the
acceleration of the laboratory, leaving only the tidal residual ---
regardless of sensor orientation. The proof requires nothing beyond
Newton's laws applied to point masses.


\section{Inertial and Non-Inertial Frames}

An \textbf{inertial frame} is one in which Newton's second law
$\vb{F} = m\,\vb{a}$ holds without correction. A body subject to no
forces moves in a straight line at constant velocity.

A \textbf{non-inertial frame} is one that accelerates. In such a frame,
Newton's laws require fictitious forces --- centrifugal, Coriolis, etc.\
--- to compensate for the acceleration of the frame itself.

\subsection*{The laboratory is non-inertial}

A laboratory on Earth's surface is non-inertial for two reasons:

\begin{enumerate}
\item \textbf{Earth rotates}, producing centrifugal and Coriolis
  pseudo-accelerations (${\sim}\,0.03$~m/s$^2$ at the equator).

\item \textbf{Earth freely falls toward the Sun and Moon.} The entire
  Earth --- lab, sensor, test mass, and all --- accelerates toward the
  Sun at $GM_S/R^2 \approx 5.93 \times 10^{-3}$~m/s$^2$ and toward the
  Moon at $GM_M/D^2 \approx 3.3 \times 10^{-5}$~m/s$^2$.
\end{enumerate}

The second point is the origin of the cancellation. When you write
equations in the lab frame, you have already subtracted these
accelerations from every term. A sensor physically implements this
subtraction: both ends of the spring accelerate identically in the
Sun's field, so the spring reads zero from that field.

To see this without hidden subtractions, we work in a true inertial
frame and carry every term explicitly.


\section{The Proof}

\subsection*{Setup}

Four point masses in an inertial frame:

\begin{center}
\begin{tabular}{lll}
\toprule
Body & Mass & Position \\
\midrule
Sun       & $M_S$ & $\vb{R}_S(t)$ \\
Moon      & $M_M$ & $\vb{R}_M(t)$ \\
Earth     & $M_E$ & $\vb{R}_E(t)$ \\
Test mass & $m$   & $\vb{x}(t)$ \\
\bottomrule
\end{tabular}
\end{center}

The test mass is connected to the sensor by a spring (or equivalent
restoring mechanism). The sensor housing is rigidly attached to the
Earth. The sensor reads the spring force~$\vb{F}_{\text{spring}}$ ---
the non-gravitational force required to keep the test mass co-moving
with the lab.

\subsection*{Step 1: Test mass equation of motion}

In the inertial frame, Newton's second law for the test mass:
\begin{equation}
m\,\ddot{\vb{x}}
  = -\frac{G M_E\, m}{\lvert\vb{x} - \vb{R}_E\rvert^3}
    (\vb{x} - \vb{R}_E)
  \;-\frac{G M_S\, m}{\lvert\vb{x} - \vb{R}_S\rvert^3}
    (\vb{x} - \vb{R}_S)
  \;-\frac{G M_M\, m}{\lvert\vb{x} - \vb{R}_M\rvert^3}
    (\vb{x} - \vb{R}_M)
  \;+\; \vb{F}_{\text{spring}}
\label{eq:test}
\end{equation}

\subsection*{Step 2: Earth's equation of motion}

Earth's center accelerates toward the Sun and Moon:
\begin{equation}
\ddot{\vb{R}}_E
  = -\frac{G M_S}{\lvert\vb{R}_E - \vb{R}_S\rvert^3}
    (\vb{R}_E - \vb{R}_S)
  \;-\; \frac{G M_M}{\lvert\vb{R}_E - \vb{R}_M\rvert^3}
    (\vb{R}_E - \vb{R}_M)
\label{eq:earth}
\end{equation}

Every object rigidly attached to the Earth --- including the sensor
housing --- shares this acceleration.

\subsection*{Step 3: Subtract to get the relative motion}

Define $\vb{r} \equiv \vb{x} - \vb{R}_E$ and the geocentric positions
$\vb{R} \equiv \vb{R}_S - \vb{R}_E$,
$\vb{D} \equiv \vb{R}_M - \vb{R}_E$. Then
$\ddot{\vb{r}} = \ddot{\vb{x}} - \ddot{\vb{R}}_E$ gives:
\begin{equation}
\boxed{
m\,\ddot{\vb{r}}
  = -\frac{G M_E\, m}{\lvert\vb{r}\rvert^3}\,\vb{r}
  \;+\; G M_S\, m \left[
    \frac{\vb{R} - \vb{r}}{\lvert\vb{R} - \vb{r}\rvert^3}
    - \frac{\vb{R}}{R^3}
  \right]
  \;+\; G M_M\, m \left[
    \frac{\vb{D} - \vb{r}}{\lvert\vb{D} - \vb{r}\rvert^3}
    - \frac{\vb{D}}{D^3}
  \right]
  \;+\; \vb{F}_{\text{spring}}
}
\label{eq:relative}
\end{equation}

\subsection*{The cancellation}

Each bracketed term has the form
\begin{equation}
\vb{f}(\vb{r})
  = \frac{\vb{R} - \vb{r}}{\lvert\vb{R} - \vb{r}\rvert^3}
  - \frac{\vb{R}}{R^3}
\end{equation}

This is the Sun's gravitational field at the test mass \emph{minus} the
field at Earth's center. At $\vb{r} = \vb{0}$:
\begin{equation}
\vb{f}(\vb{0})
  = \frac{\vb{R}}{R^3} - \frac{\vb{R}}{R^3}
  = \vb{0}
  \qquad\text{(exactly)}
\label{eq:cancel}
\end{equation}

The direct pull --- $GM_S\,\vb{R}/R^3$, the uniform field that
accelerates the entire Earth at $5.93 \times 10^{-3}$~m/s$^2$ ---
appears with equal magnitude and opposite sign in
Eqs.~\eqref{eq:test} and~\eqref{eq:earth}, and cancels identically.
No approximation is involved. The same holds for the Moon with $\vb{D}$
replacing~$\vb{R}$.

What survives is $\vb{f}(\vb{r}) - \vb{f}(\vb{0})$: the
\emph{variation} of the gravitational field across the
baseline~$\vb{r}$ --- the \textbf{tidal acceleration}. To leading order
in $r/R$:
\begin{equation}
\boxed{
\vb{a}_{\text{tidal}}
  = \frac{GM}{R^3}\Big[
    3\,(\vu{R}\cdot\vb{r})\,\vu{R} - \vb{r}
  \Big]
  + O\!\left(\frac{GM\,r^2}{R^4}\right)
}
\label{eq:tidal}
\end{equation}

This is suppressed by a factor $r/R$ relative to the direct pull.

\subsection*{Derivation of Eq.~\eqref{eq:tidal}}

Expand $\lvert\vb{R} - \vb{r}\rvert^{-3}$ for $r \ll R$. Write
\begin{equation}
\lvert\vb{R} - \vb{r}\rvert^2
  = R^2\!\left(1
    - 2\,\frac{\vu{R}\cdot\vb{r}}{R}
    + \frac{r^2}{R^2}\right)
  \equiv R^2(1 - \epsilon)
\end{equation}
with $\epsilon = 2\,\vu{R}\cdot\vb{r}/R - r^2/R^2 = O(r/R)$. Then
\begin{equation}
\frac{1}{\lvert\vb{R} - \vb{r}\rvert^3}
  = \frac{1}{R^3}(1-\epsilon)^{-3/2}
  = \frac{1}{R^3}\!\left(1
    + \frac{3\,\vu{R}\cdot\vb{r}}{R}
    + O\!\left(\frac{r^2}{R^2}\right)\right)
\end{equation}

Multiply by $(\vb{R} - \vb{r})$ and keep terms through first order:
\begin{equation}
\frac{\vb{R} - \vb{r}}{\lvert\vb{R} - \vb{r}\rvert^3}
  = \frac{1}{R^3}\Big[
    \vb{R} + 3(\vu{R}\cdot\vb{r})\,\vu{R} - \vb{r}
  \Big]
  + O\!\left(\frac{r^2}{R^4}\right)
\end{equation}

Subtract $\vb{R}/R^3$:
\begin{equation}
\vb{f}(\vb{r})
  = \frac{\vb{R} - \vb{r}}{\lvert\vb{R} - \vb{r}\rvert^3}
  - \frac{\vb{R}}{R^3}
  = \frac{1}{R^3}\Big[
    3(\vu{R}\cdot\vb{r})\,\vu{R} - \vb{r}
  \Big]
  + O\!\left(\frac{r^2}{R^4}\right)
\end{equation}

The $\vb{R}/R^3$ terms cancel identically --- confirming that the
direct pull drops out --- and the leading surviving term is the tidal
quadrupole, Eq.~\eqref{eq:tidal}.\hfill$\square$


\section{Transformation to the Laboratory Frame}

Equation~\eqref{eq:relative} is written in the non-rotating geocentric
frame. The laboratory rotates with the Earth at angular
velocity~$\boldsymbol{\omega}$. This is a second source of
non-inertiality (see Section~1).

\subsection*{Rotating-frame equation of motion}

Let $\vb{r}'$ be the test mass position in the lab frame, related to
its geocentric position by
$\vb{r} = \mathcal{R}(t)\,\vb{r}'$
where $\mathcal{R}(t)$ is the time-dependent rotation matrix.
Differentiating twice:
\begin{equation}
\ddot{\vb{r}}
  = \ddot{\vb{r}}'
  + 2\,\boldsymbol{\omega} \times \dot{\vb{r}}'
  + \boldsymbol{\omega} \times (\boldsymbol{\omega} \times \vb{r}')
\label{eq:kinematic}
\end{equation}
(the Euler term $\dot{\boldsymbol{\omega}} \times \vb{r}'$ vanishes for
uniform rotation). Substituting into Eq.~\eqref{eq:relative} and
expressing all vectors in the lab basis:
\begin{equation}
\boxed{
\ddot{\vb{r}}'
  = \underbrace{-\frac{G M_E}{\lvert\vb{r}'\rvert^3}\,\vb{r}'
    }_{\text{Earth's gravity}}
  \;+\; \underbrace{\vb{a}_{\text{tidal}}'
    }_{\text{tidal}}
  \;\underbrace{-\; \boldsymbol{\omega}
    \times (\boldsymbol{\omega} \times \vb{r}')
    }_{\text{centrifugal}}
  \;\underbrace{-\; 2\,\boldsymbol{\omega}
    \times \dot{\vb{r}}'
    }_{\text{Coriolis}}
  \;+\; \frac{\vb{F}_{\text{spring}}'}{m}
}
\label{eq:lab}
\end{equation}

The direct solar and lunar pulls do \textbf{not} reappear --- they
canceled in Eq.~\eqref{eq:relative} before the frame change, and
rotating the coordinate basis cannot restore a term that is already
zero.

\subsection*{What the sensor reads}

The test mass sits at rest in the lab:
$\dot{\vb{r}}' = \ddot{\vb{r}}' = \vb{0}$. The Coriolis term
vanishes. Setting the left-hand side of Eq.~\eqref{eq:lab} to zero:
\begin{equation}
\frac{\vb{F}_{\text{spring}}'}{m}
  = \frac{G M_E}{\lvert\vb{r}'\rvert^3}\,\vb{r}'
  + \boldsymbol{\omega} \times (\boldsymbol{\omega} \times \vb{r}')
  - \vb{a}_{\text{tidal}}'
\label{eq:spring}
\end{equation}

The sensor projects this onto its measurement axis~$\vu{n}$:
\begin{equation}
g_{\text{measured}}
  = \frac{\vb{F}_{\text{spring}}'}{m} \cdot \vu{n}
\end{equation}

The three contributions are:
\begin{itemize}
\item $G M_E\,\vb{r}'/\lvert\vb{r}'\rvert^3$: Earth's gravity (the
  dominant, static term).
\item $\boldsymbol{\omega}\times(\boldsymbol{\omega}\times\vb{r}')
  = -\omega^2\vb{r}'_\perp$: the centrifugal reduction (objects weigh
  less at the equator by ${\sim}\,0.3\%$).
\item $-\vb{a}_{\text{tidal}}'$: the tidal perturbation from Sun and
  Moon, of order $10^{-7}$~m/s$^2$.
\end{itemize}

\noindent
The direct pull $GM_S/R^2$ and $GM_M/D^2$ appear nowhere. This holds
for any~$\vu{n}$.

\subsection*{Size of each term}

\begin{center}
\begin{tabular}{lll}
\toprule
Term & Expression & Magnitude \\
\midrule
Earth's gravity
  & $GM_E/r_\oplus^2$
  & $9.81$~m/s$^2$ \\
Centrifugal (equator)
  & $\omega^2 r_\oplus$
  & $3.4 \times 10^{-2}$~m/s$^2$ \\
Lunar tide
  & $GM_M\, r_\oplus / D^3$
  & $1.1 \times 10^{-6}$~m/s$^2$ \\
Solar tide
  & $GM_S\, r_\oplus / R^3$
  & $5.1 \times 10^{-7}$~m/s$^2$ \\
Solar direct pull (canceled)
  & $GM_S / R^2$
  & $5.93 \times 10^{-3}$~m/s$^2$ \\
\bottomrule
\end{tabular}
\end{center}

\noindent
The sensor reads a superposition of the first four terms. The fifth ---
the direct pull --- has been subtracted out by the physics of the
measurement: the lab accelerates with the Earth, and the spring cannot
see it.


\section{Why Sensor Orientation Is Irrelevant}

The cancellation in Eq.~\eqref{eq:cancel} is \textbf{vectorial}: the
direct pull vanishes as a three-component vector, not as a particular
scalar projection. Projecting onto any measurement axis~$\vu{n}$:
\begin{equation}
\vu{n} \cdot \vb{f}(\vb{0}) = \vu{n} \cdot \vb{0} = 0
  \qquad\text{for all } \vu{n}
\end{equation}

Tilting the sensor changes~$\vu{n}$ but cannot make zero non-zero.
The spring force along any axis contains only the tidal terms from
Eq.~\eqref{eq:tidal}.

Physically: the sensor housing and the test mass both accelerate at
$GM_S/R^2$ toward the Sun. The spring connecting them measures only the
\emph{difference} between the accelerations of its two endpoints. A
uniform field produces no difference, regardless of the spring's
orientation.

\subsection*{Sweeping the sensor orientation}

Parameterise the measurement axis by zenith angle~$\theta$ and
azimuth~$\varphi$:
\begin{equation}
\vu{n}(\theta,\varphi)
  = \sin\theta\cos\varphi\;\vu{E}
  + \sin\theta\sin\varphi\;\vu{N}
  + \cos\theta\;\vu{U}
\end{equation}
where $\vu{E}$, $\vu{N}$, $\vu{U}$ are the local East--North--Up
unit vectors defined relative to the \emph{geodetic} vertical ---
the plumb-line direction, which already includes the centrifugal
deflection from the geocentric radial. (The centrifugal acceleration
$-\omega^2\vb{r}'_\perp$ points away from the rotation axis, not
along the geocentric radial, so it has both vertical and
horizontal components in a geocentric basis. The geodetic basis
absorbs this into the definition of ``Up.'')

In this basis, the effective gravity
$\vb{g}_{\text{eff}} = (GM_E/r_\oplus^2)\,\hat{\vb{r}}
  - \omega^2\vb{r}'_\perp$
is purely along~$\vu{U}$ with magnitude~$g_0$. From
Eq.~\eqref{eq:spring}, the measured acceleration as a function of
orientation is then
\begin{equation}
g(\theta,\varphi)
  = g_0\cos\theta
  \;-\; a_E'\sin\theta\cos\varphi
  \;-\; a_N'\sin\theta\sin\varphi
  \;-\; a_U'\cos\theta
\label{eq:sweep}
\end{equation}
where $g_0 \approx 9.81$~m/s$^2$ (varying with latitude) and
$a_E'$, $a_N'$, $a_U'$ are the East, North, and Up components of the
tidal acceleration in the geodetic basis.

Sweeping~$\theta$ and~$\varphi$ traces out the full angular
dependence. If the direct pull $GM_S/R^2$ were present, it would
contribute a term $\propto \sin\theta\cos(\varphi - \varphi_\odot)$
with amplitude ${\sim}\,6 \times 10^{-3}$~m/s$^2$. Instead, the
horizontal terms carry amplitudes
$a_E', a_N' \sim 10^{-7}$~m/s$^2$ --- the tidal values --- and
the vertical tidal correction~$a_U'$ is of the same order.

An angular scan therefore confirms the cancellation directly:
one measures $g_0\cos\theta$ plus tidal corrections of order
$10^{-7}$~m/s$^2$, with no $10^{-3}$~m/s$^2$ sinusoidal component
at any orientation.


\section{Numerical Scale}

\begin{center}
\begin{tabular}{llll}
\toprule
Body & Direct pull (cancels) & Tidal residual (observable) & Ratio \\
\midrule
Sun  & $5.93 \times 10^{-3}$~m/s$^2$
     & $5.1 \times 10^{-7}$~m/s$^2$
     & $11{,}700\times$ \\
Moon & $3.32 \times 10^{-5}$~m/s$^2$
     & $1.1 \times 10^{-6}$~m/s$^2$
     & $30\times$ \\
\bottomrule
\end{tabular}
\end{center}

The Sun's direct pull is $180\times$ stronger than the Moon's, but its
tidal effect is $2.2\times$ \emph{weaker} ($1/R^3$ vs.\ $1/R^2$
scaling). This inversion is the hallmark of tidal physics, and a direct
consequence of the cancellation.


\section{What IS Observable}

Tilting the sensor accesses different components of the \emph{tidal}
field, not the direct pull:

\begin{itemize}
\item A vertical sensor sees the vertical tidal component
  (${\sim}\,5 \times 10^{-7}$~m/s$^2$, semi-diurnal).
\item A horizontal sensor picks up horizontal tidal components
  (diurnal).
\item Three non-coplanar sensors reconstruct the full tidal tensor.
\end{itemize}

These effects are real and measurable, but four orders of magnitude
smaller than the claimed signal.

\end{document}
